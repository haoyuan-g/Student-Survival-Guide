\documentclass[11pt,oneside]{book}
\usepackage{ctex}
%%%%%%%%%%%%% Geometry
\usepackage[a4paper,left=2.5cm,right=2.5cm, bottom=2.5cm,top=2.5cm]{geometry}
%%%%%%%%%%%%%%% Les paquets
\usepackage[english]{babel}
\usepackage[palette=munch]{nexus}
%%%%%%%%%%%%%%%% hyperref
\usepackage[verbose]{hyperref}
\usepackage{titlesec}
\usepackage{graphicx}

\hypersetup{ 
    hidelinks
}
\setlength{\XeTeXLinkMargin}{-1pt}
%\setlength{\lineskip}{2.625bp}

%首行缩进
\usepackage{indentfirst}
\setlength{\parindent}{2em}

%目录仅显示第0级和第1级
\setcounter{tocdepth}{1}

% 正文从这里开始
\begin{document}

\pagestyle{empty}

\definecolor{plop}{HTML}{4D7186}
\begin{textblock}{1}(0,0)
    \noindent\textcolor{plop}{\rule{\paperwidth}{.55\paperheight}}
\end{textblock}


\begin{textblock}{1}(0,.55)
    \noindent\textcolor{black}{\rule{\paperwidth}{.45\paperheight}}
\end{textblock}


\begin{textblock}{1}(.1,.09)
    \noindent{\fontsize{24.88}{2}\selectfont
        \bfseries\textcolor{white}{北京大学工学院}}
\end{textblock}

\begin{textblock}{1}(.1,.15)
    \noindent {\fontsize{24.88}{2}\selectfont
    \bfseries\textcolor{white}{学生生存手册}}
\end{textblock}

% \begin{textblock}{1}(.1,.21)
%     \noindent{\fontsize{30}{2}\selectfont
%         \bfseries\textcolor{white}{for \LaTeX}}
% \end{textblock}

\begin{textblock}{1}(.1,.45)
    \noindent {\fontsize{20.74}{2}\selectfont
        \bfseries\textcolor{white}{工学院学生会学术部}}
\end{textblock}



\begin{textblock}{.9}(.05,.56)
    \begin{flushright}
        \noindent {\fontsize{20.74}{2}\selectfont
            \bfseries\textcolor{orange}{version 1.0}}
    \end{flushright}
\end{textblock}


\begin{textblock}{.45}(.5,.82)
    \begin{center}
        \includegraphics[width=.45\paperwidth]{dlsin}
    \end{center}
\end{textblock}

\begin{textblock}{.4}(.05,.65)
    \begin{center}
        \includegraphics[width=.4\paperwidth]{arccos}
    \end{center}
\end{textblock}


\begin{textblock}{.6}(.05,.6)
    \noindent {\fontsize{20.74}{18}%
    \textcolor{white}{$\displaystyle(a+b)^n = \sum_{k=0}^n 
                \binom{n}{k} a^kb^{n-k}$}}
\end{textblock}


\begin{textblock}{.4}(.4,.77)
    \noindent {\fontsize{17.28}{18}%
    \textcolor{white!80}{$\displaystyle 
                \neg (p\vee q) \equiv (\neg p)\wedge (\neg q)$}}
\end{textblock}

\begin{textblock}{.4}(.1,.93)
    \noindent {\fontsize{14.4}{18}%
    \textcolor{white!50}{$\displaystyle 
                \binom{n}{k} = \frac{n!}{k!(n-k)!}$}}
\end{textblock}


\begin{textblock}{.6}(.5,.69)
    \noindent {\fontsize{17.28}{18}%
    \textcolor{white!10}{$\displaystyle 
                \zeta_k = |a|^{1/n} \mathrm{e}^{i(\mathrm{arg}(a)+2k\pi)/n}$}}
\end{textblock}


\begin{textblock}{.3}(.75,.73)
    \noindent {\fontsize{17.28}{18}%
    \textcolor{white!10}{$\displaystyle \mathrm{e}^{i\pi}+1=0$}}
\end{textblock}



\null\newpage\pagestyle{nexus}


\frontmatter
\chapter*{前言}
那是一个寒冷的初春。一条不起眼树洞上的几条回复几乎可以说直接导致了这份手册的产生。2024年3月,工学院学生会学术部开始对本手册进行筹划。经过一个暑期的编写,新生版的手册终于能够呈现在初入燕园的同学们手中。

笔者首先感谢学术部倪昊、武昱达同学抽出暑假时间来编写这份手册,感谢学院内各位老师们的悉心指导和谢谊锋、吴秉宪以及手册中约稿内容的作者们等诸多同学的大力支持;没有他们的努力和智慧,这份手册不会如此快速、高效地呈现在大家面前。笔者也要感谢入校以来与自己沟通、交流过手册编写的所有人,与他们的交流为手册提供了灵感和准备。在这里,我们同样期待着每一位愿意向大家传递信息、分享经验,让本手册更好、更充实的各位加入我们。更新迭代是手册的必经之路,只有这样,它才能跟上时代的发展,才能更好地为更多同学的成长与发展服务。

回到最初,是什么引发了我们编写本手册的想法?我回顾在工学院度过的这两年,深感学院在经历剧烈的变化,从招生人数的剧增、培养方案的改写、教师队伍的“握手计划”、新奥大楼的建成等等方面均可见一斑。招生人数的增加本应该让同学之间的相互交流学习变得更加轻松,但我看到了很多同学没有办法从现有的渠道中获得信息,也不愿意找老师、身边的同学进行交流,最终导致错过了很多机会;我也看到了有些同学之间的陌生、小范围抱团却不免闭塞的交流圈子;我还注意到老师们、学长学姐们的分享往往是点对点的,会消耗大量不必要的时间精力,且只能覆盖问题在个体上的投射,需要多个角度才能知晓全貌……我始终认为:从我入学直至现在两年的时间,工学院同学们在信息获取和交流方面的低效状况未曾有大的改变,却亟待解决。本手册希望做出一些力所能及的贡献。

当然手册的任务也不仅如此。手册还希望帮助大家在本科四年里寻求一些问题的答案:我是怎样的人?我喜欢什么?我想要去做些什么?新的时代快速发展的浪潮下,这些问题是需要想清的,因此我们可以尽早开始思考。

最后笔者想向大家明确一点:本手册不是《高分保研奖学金捷径手册》或者《内卷宝典》。本手册希望帮助大家的,决不是仅仅拿一个高高的GPA(Grade Point Average,平均学分绩点)、发SCI(Science Citation Index,科学引文索引)论文、收到顶尖学校的直博录取等等诸如此类能被量化的指标。工学院的课程压力不小,但生活不能只有专业课和成绩,燕园里的其他风景也同样精彩。我希望这份手册能够带给大家一点恰当的提示、一些新鲜的思考,让同学们能看到人生道路上不同的风景。至少在读过本手册之后,各位应该能够有勇气和智慧,去发现并挑战一些比既定指标更重要的追求。
\begin{flushright}
    赵丹枫
    
    2024年8月
\end{flushright}

\chapter*{声明}

《北京大学工学院学生生存手册》是一份由学生自发策划、组织编写的手册,版权属于北京大学工学院学生会。未经北京大学工学院学生会书面许可,任何组织或个人不得违反相应版权条例抄袭、转载、摘编、修改本书内容;不得将本书用于商业目的;不得对本手册原意进行曲解、修改和未授权的大范围分发。对于未经授权传播手册全部或部分内容而造成的各种问题,手册编者概不负责。

该手册集合众多学长学姐的经验和观点,并尽可能将各种观点统一在手册的框架下。这样做的目的,是希望向读者传达更多可供参考的观点和意见。请读者注意:本手册不构成任何明确的行动建议,不保证手册中的信息和方法始终正确、有效。编者不承担由手册及其内容产生的衍生责任。
\begin{flushright}
    北京大学工学院学生会

    2025年8月
\end{flushright}

\chapter*{新生版前言}
开辟鸿蒙,谁为学种?都只为工学情浓。趁着这迎新天、报到日、欢乐时,试着剧透。因此上,撰写这震古烁今的《生存册》。——《工学梦曲》

欢迎各位新同学进入北京大学工学院!

《北大工学院本科生生存手册》(以下简称“手册”)的撰写是一个长期项目,由工学院学生会全权负责。考虑到手册编写的目的是尽可能及时造福同学们,我们决定,1.0版本的手册面向的群体是本科新生,只会撰写最核心、最基础的一些内容,加之一些新生需要知道的事宜,以帮助刚刚入学的各位coeer快速弥合信息差,拓宽知识面,感受工院文化,找到属于自己的生活和学习节奏。

本手册旨在汇集学院老师、学长学姐们的经验和智慧,以一种自由且简洁的方式,为大家呈现直接、真实、系统性的工学院学习生活重要信息,减少大家搜索信息、辨别信息、整理信息的时间消耗,把更多的时间精力投入到对自己生涯的思考和规划、具体知识的学习,和有意义的大学活动中,收获一段真正充实而精彩的大学生涯。当然,我们相信大家想要了解的信息不止于手册中包含的内容,因此我们在手册最后提供了问卷二维码,欢迎同学们进一步提问,或对我们的工作提出更好的建议。

手册自2024年3月开始筹备编写,如今尚处于“幼年”。希望同学们能从手册中获得一些启发,也希望更多的同学积极加入工学院学生会,为手册的编写贡献自己的智慧和力量!
\begin{flushright}
    倪昊

    2024年8月
\end{flushright}




\tableofcontents

\mainmatter

\chapter{基本情况}
\section{小而精的北大工科}
2005年我们建立工学院的时候,招了一批最顶尖的学者,也对工学院有很多期许。然后我们就面对这个问题——我们北大要做什么样的工科?这个工学院是要做什么样的工科?当时工科有两种模式,一种是MIT的,它是那种大而全的工科;另一种是Caltech模式,这是钱学森的母校,它是一个体量很小的学校,但做出了相当出色的成果。

于是我们决定走Caltech的道路,做出小而精的工科,做最前沿的、对未来可能有影响的东西。当时提出的说法是“science based engineering”,这个说法至今我们还在用。我的理解是,我们说engineering,就是说我们要把一个东西造出来,但这个science,强调的就是不要按别人的路造出来,而是要创新性地、从0到1地做出新的东西。
\begin{flushright}
    ——摘自工映青春公众号,唐少强老师访谈《从零开始的力学生活》

\end{flushright}

近几年学院仍然贯彻了“小而精”的理念,老师们做的方向各有千秋,有学科主流方向,但更多是一些学术界内做的相对较少的方向。作为工学院建院基础的力学是基础学科,在现在这个时代要想取得长足的发展,必然需要进行学科交叉;这也是学院目前各学科发展的大趋势。

\section{工学院的新鲜血液}    
\subsection{招生人数与人员分布}
笔者全程参加了今年北京大学的招生工作,明显感受到今年理工科名额增加,以及来咨询工学院的学生、家长突然变多,这也是学院发展、招生政策、经济形势、社会环境不断变化所共同决定的。工学院2008至2023年本科生招生人数见下图。近三年数据为:2021级141人(其中力学类强基生95人)、2022级155人(其中力学类强基生101人)、2023级本科生184人(其中力学类强基生124人)。可以看出,近年来工学院本科生招生人数稳步上升。

\begin{figure}[htbp]
    \centering
    \includegraphics[width=0.6\linewidth]{1.2.1人数图.png}
    \renewcommand{\figurename}{图}
    \caption{工学院近年本科生招生人数}
    \label{fig:enter-label}
\end{figure}
        
大一末分专业方向的时候,每届的同学选择也都非常不同。例如2019级,新设立机器人工程方向,50人左右选择了这个方向;2022级则理论与应用力学方向人数最多;2023级对生物医学工程方向的热情又有所提高。

所以,笔者想要指出的是:学生的绝对数量和人员组成的年际变化非常剧烈。学生绝对数量的增加很可能加剧院内竞争压力,且从2021到2023级看来,这个趋势已经较为明显。更细致地,人员组成的变化会导致每一个专业方向当中,学生数量、教师精力等等诸多方面的不匹配。这可能会在今后的学习生活中给大家带来无法预料的影响,我们需要向大家提示这一点。
        
\subsection{工科试验班与强基计划}
工学院主要从两条轨道录取新生:工科试验班(此类同学后称“工试同学”)和强基计划(此类同学后称“强基同学”)。这两类学生面临的政策有部分差异:

第一,院系转换。强基计划原始文件中规定:经由强基计划录取的学生原则上不能转专业。但在这一点上北大校内政策相对宽容:强基计划专业内部可以互转。工试同学原则上没有转专业限制。

第二,推荐免试攻读研究生。工试同学走传统的保研路径,强基同学走转段路径。强基同学自带转段名额(转段名额什么条件下被取消自行查找),而GPA在推免线上的工试同学才能获得推免名额。但不管工试同学还是强基同学,推免之后找老师接收这一步都需要自己进行。在接收方,强基同学还有一个限制:强基计划有委托培养性质,也就是说北大的强基生只能保送本校攻读研究生,而不能是其他学校;而工试同学可以找其他院校的老师接收自己。此外,强基计划和工科试验班的同学都不限制考研以及出国攻读研究生。

第三,培养方案。详见教务部官网《北京大学本科教学计划》或工学院官网上的培养方案,其中强基与工试培养方案分列。

以上提到的事项(尤其是强基同学)每年具体政策都可能有变化,所以希望大家多多关注相关消息。

 
\section{细谈本科专业}
以下介绍工学院的七大本科专业方向,主要采用向高年级学长学姐约稿介绍的形式,内容仅供参考。尤其是与培养方案和修读课程相关的问题,稿件中学长学姐们都以2023级及之前的培养方案作为参考。请同学们仔细阅读,独立思考,不理解或与现行方案有出入之处还请咨询老师或学长学姐。
\subsection{理论与应用力学}
\subsubsection{(1)2021级\ 谢谊锋}
\textbf{P1 理力专业学习内容}

这里分为四个部分讲解:

\textbf{第一部分为政治课、英语课、体育课、通识课等课程},这一部分课程可以有三种心态进行学习:完成毕业学分,因为北大本科毕业要求学生必须修完这类课程,所以可以选择任务量较少的课程,满足毕业要求;自己感兴趣的方向内容,因为北大在这些课程开设具有很大的广度,所以如果有正好在兴趣点上的课程,可以进行选取;通过这类课程提高GPA,因为目前对于非强基生,保研的推免资格和总GPA相关。并且总GPA越高,在学年的评奖评优中也有优势。但是工学院是较为重视专业课的学习成绩的,所以对于这部分课程,同学们不应选投入过多精力的课,而可以选修一些给分较好的课程(当然,也会比较难抢上)。

上述三种心态可能同时存在一门课程中,这当然是比较理想的情况。但是哪怕不感兴趣,不想在这类课程上花太多心思也当然是在工院同学中普遍存在的心理。
 
\textbf{第二部分为数学、力学、计算机的基础专业课},下面列举了部分课程:
\begin{itemize}
    \item 数理基础课程:\\数学分析((一)(二)(三)、线性代数、高等代数、常微分方程、数学物理方法(上)(下)、概率论、数理统计……
    \item 力学基础课程:\\理论力学、材料力学、弹性力学、流体力学……
    \item 计算机基础课程:\\计算概论、数据结构与算法、计算方法……
\end{itemize}

这类课程的具体工程应用性不强,相反,课程强调的是理论与概念,对于部分同学来说不是那么直观,甚至是难以理解的。但是希望同学们能够认真地学习这部分专业课,因为这部分知识,对于未来进入到相关的需要数理知识的应用场景中,会有极强的核心竞争力。

同时,学院也是很重视这一部分的学习,所以对于未来希望升学进修的同学,这一部分的学习成绩、学习情况是极为关键的。

并且,在我的感受中,我是比周围人理解能力要差一点的人,弄明白一些抽象概念花的时间比优秀的同学要多一点,但是对于工学院的课程,只要肯下功夫,都是能够弄明白的,所以同学们也需要有自信心。

\textbf{第三部分为自主选修课},这类课程同学们可以在全校范围内选课(但是需要注意,部分课程是不计入自主选修课学分),因此这部分学分是否能被计入需要和教务进行确认。大家可以根据自己的兴趣,科研要求进行更加针对性的学习。

\textbf{第四部分是工学院的实验课,基础物理课程}。这类课程和工学院专业的关联性较弱,但是实际应用性很强,并且能够锻炼同学们的数理思维,也建议用心学习。

\textbf{P2 专业未来}

本科就业:优势很弱,因为同学们的大量精力花费在了抽象的数理知识的学习上,对于具体的工程应用经验其实是缺乏的,并且本人身边同学在理力方向几乎没人抽出精力去企业进行实习。并且,本科学习阶段,数理基础是比较泛的,比如在固体力学这一块学到弹性力学便结束了,对于更深层次的塑性力学、断裂力学等课程的知识是没有掌握的,所以想进入到某一方向的头部就业岗,除了教育培训,其余对口率都很低。此外,本人也接触过某里、某方等头部企业的研发岗,我认为其工作内容对于我们学习到的这些抽象数理概念利用率低,而工作重心是在满足甲方要求的前提下,用最低的成本,最简洁明了的,最安全稳定的,最高效速成的方法完成甲方要求的项目。当然,北大文凭还是很强的敲门砖,并且同学们学习能力也很强,这是我们的优势区间。

升学科研:这是工学院大部分同学的选择,同学们通过在工学院本科阶段的学习,正如我前面提到学习的科目是很泛的,所以同学们是很可能找到和自己兴趣、能力相匹配的科研方向的。并且在这一阶段,可以将之前学习到的数理知识这一“内功”练就为一门独特的“外功”,将核心竞争力培养得更加具体化。
 
\textbf{P3 学习体会}

成熟的培养方案:在力学系悠久的历史中,本培养方案中数理课程甚至是北大很多任教老师上过的课程,也是很多优秀毕业学长学姐们上过的,因此经过时间检验这一培养方案在培养数理基础的认知上是极为成熟的。同时,这一培养方案与具体内容也是不断革新的,保证了同学们可以在掌握基础的同时,对于力学相关前沿问题也能够有更加全面的认知视野。

学习过程是十分辛苦的:无论是本科生还是研究生阶段,力学系的学习都是需要同学们花费大量时间和精力的。相比其他方向的同学,力学系的卷度是更高的,比如力学系同学希望达到较好的成绩或专业排名,则需要花费极大的精力。即使是我这样大多数数理课程在优秀以上,自己擅长的数理课程能拿90+的同学,也是极为刻苦的,并且以我对于头部同学的观察,他们也更加用功。

对于力学数理知识有敬畏之心,对于其中知识抱有好奇之心,并以在其中探索为乐趣的同学,你们的付出是可以得到回报的:我认识的大部分同学,包括我在内,都认为我们在数理基础上投入的时间精力是得到了很强的幸福感的回报的,但是具体内容是因人而异的,所以这一部分我不具体展开描述,就留给同学们自己去发现吧。

\subsubsection{(2)2021级\ 杨铮昊}

传统的力学专业分流体和固体两派,我们最早会在大二下学期接触到的《材料力学》就是固体力学的范畴,而直到大三我们才会通过两门必修课《流体力学(上)》《流体力学(下)》入门流体力学这门学科。我挺建议对固体力学不是很感冒的同学提前了解一下流体力学的内容,或许你就能更早发现自己的兴趣其实在流体。除了这两派以外,工院的力学与工程科学系还包括了许多其他的力学分支如生物力学、力学系统与控制,也包括一些交叉学科如流固耦合。这些方向需要学习的内容都不会出现在培养方案的推荐课程内,所以建议想要探索的同学尽早去联系对应的老师寻求指导和帮助。

理力专业的课程方面恰如其名,围绕着“理论”与“应用”两个主题。在低年级优先学习的数学课和力学课都是为了打下扎实的“理论”基础,而大二下学期的《材料力学实验》才正式开启“应用”的实践。如果学习理论课程时没有那么得心应手,不妨多去尝试一些实验课。这里的“实验”并不仅限于实验室里的实验,也包括那些需要动手编程的数值模拟实验。因为在未来的科研中,真实实验和数值实验都是推动科研的重要实验手段。

理力专业的课程难度总是被排在第一,因为比其他专业多出一些“更难”的数学课。但对于大多数专业而言,要想行稳致远都离不开扎实的数理基础,而理力专业只是更多强调了这一点且体现在了课程设置上。所以无论是选择哪个专业,基础数学课都一定要认真学习,力求扎实掌握。

有玩笑说各行各业都有理力专业的学生,这其实不假。通过本科阶段的学习,理力专业的学生应该有信心能收获较为扎实的数理基础和动手能力,今后无论是在哪个领域从事科研或是就业,二者都能提供充足的竞争力。另一方面,如果你能够尽早地确定未来的奋斗方向、尽快入门并启动,更是一件好事。所以本科阶段除了沉淀和积累,很重要的一点就是多向外探索。

作为在此专业学习已达三年之久的“老登”,我还是感觉很幸福的,因为无论是老师还是同学都特别友善包容,让我既能在良好的学习氛围中收获满满,又敢于做出许多探索和尝试。但如果这个专业不适合你,同样放心寻求老师的帮助就好。

\subsubsection{(3)周培源班}

周培源书院(或周培源班)是工学院理论与应用力学专业方向专门开设的一个小班。有关招生人数、招生时间(请有意向的同学们仔细关注学院官网上的报名通知,不要错过时间)等基本信息,大家在12月学院举办的工学嘉年华、大一下学期举办的若干次选专业指导讲座中可以获悉。

以下是周班的学长给大家作的补充介绍:

\textbf{周班介绍$\parallel$作者:2021级\ 钱骏飞}

\textbf{\textbf{一、发起周班的动机及培养目标:}}

秉承“培本固源,笃行秀出”的理念,旨在培养出具有深厚数理基础和人文素养、追求长远科学目标、能够引领力学学科及未来新技术创新发展的杰出人才。

\textbf{\textbf{二、周班与普通班培养的差别:}}

目前周班同学接受的培养方案和普通班是完全一致的。在课程上的差别是,周班同学接受的是小班化教学,授课内容会略多于平行班,课程的考核难度也会较高。考虑到周班的同学们都很优秀,课程都是不限优秀率的。不过客观来说,给分并不是很香。如果大家希望的是更好地打牢自己的数理基础,多和身边优秀的老师同学们交流,对绩点不是特别care的话,非常建议大家来周班进行学习;但如果之后有出国的意愿,或者对绩点看得比较重的话,一定要慎重选择,在周班想取得特别高的成绩肯定是要比普通班难好多的。

除此之外呢,周班的一大优势就是灵活。每隔一段时间(大约是一学期一次左右吧),周班的主管老师们会和周班同学们进行线上交流,询问一下同学们目前的学习情况,以及对课程进度的一些看法。如果建议得当的话,很有可能会进行一定的课程改革,比如将某些课提前一学期学或者缩短or增长学时。在选课方面,周班是可以进行课程替代。比如国家精品课“基础物理实验”,就可以通过选修数院的实变或者微分几何来替代。原则上说,必须要选取难度更高的课才能进行替代。至于具体到哪门课可以用哪门课来替代的话,最好要向周班主管老师荣起国老师询问确认好,并且经过教务同意。整个流程操作起来虽然略显复杂,但是真的很有用!

学习之余,活动也是不少滴!周班会为大家尽可能地安排讲座、报告等活动,带领同学们充分了解科技前沿进展,帮助大家认清未来可能从事的科研方向。假期会不定期组织同学们出国参观,并进行学术等各方面的交流,例如今年五一去了马德里和巴黎,暑假去德国等等。综上所述,课余生活还是很舒服的!

进入周班的门槛并不高,只要大家数理基础别太烂(会有个面试,别被老师拷打得太狼狈一般就没问题),愿意接受这种培养方式,基本就能成为周班大家庭的一员。

\textbf{\textbf{三、周班的转入转出机制}}

对普通班的同学,想在周班选拔之后进入周班是完全可以的。需要提前跟周班的主管老师们联系好,征得同意后达到一定的条件即可转入。周班的同学们倘若迫于课业压力而想转去平行班的话也是可以的,跟老师们提前沟通好就行。

值得注意的一点是,普通班的同学是可以修周班的小班课来替代普通班的课,但周班的同学是不可以修普通班的专业课来替代周班的专业课的,一旦如此操作便会被视为退出周班。


\subsection{工程与科学计算\&工程力学}
\subsubsection{(1)2021级\ 吴秉宪}
            
在我的理解中,工算方向的教学目标是培养一批兼具数学、力学基础和计算机编程能力的同学,初衷是为工业软件的设计和工程领域的计算任务积累后备力量。因此,工算的培养方案同时包括了数学分析、线性代数、高等代数、常微分方程等数学基础课,理论力学、材料力学等力学基础课,计算流体力学、计算固体力学等结合了计算背景的力学课,以及一系列与程序设计或优化方法有关的课程。

下面我仅站在个人立场分享我在工算方向学习三年以来的一些心得。

工算的课程涉及的范围较广,很多课程之间的关系不大,例如并行程序设计、计算机图形学、工程CAD、机器学习基础等,每门课基本都是一门全新的学问。这一点容易让人感到矛盾。一方面,我们能在本科阶段接触到计算机在不同子领域的应用,对完全不同的技术都有所涉猎,便于探索自己感兴趣的方向;另一方面,课程的广度不可避免地造成深度的欠缺,每门课可能只能帮助我们窥见一角,想要深入挖掘还是需要自己在课外花额外的时间,而且课程的广度意味着培养方案前后衔接得不十分紧密,每门课想学好都得花足够的时间。对于我来说,培养方案中的肯定是存在屑课的(这就需要灵活运用课程测评来避雷了,樂),但我并不介意这样的培养模式,本科期间能够多尝试一些方向是一件好事。

那么什么样的同学适合选工算?我们当时分流时,除了理力方向占绝对优势外,最受欢迎的应该就是工算了,我想这可能是因为那时候计算机热潮仍然占据主流。随着现在新能源、生医工、新材料的需求和热度升高,近两年几个小方向的人数几乎都差不多了。我认为在这样的趋势下选专业时,自己的生涯规划是最重要的。对于有明确目标或者兴趣要去某个行业的,就不必再看接下来的建议,只管在契合的领域大胆冲就是了,因为工学院不同方向之间的差异很大,本科就提前接触自己感兴趣的方向肯定更好,也更容易寻求升学的机会。对于没有明确兴趣的同学,我想从培养方案的角度给一些建议。如上所述,工算的培养方案会包括很多和计算机相关的课程,这些课虽不像信科的专业课那么硬核,不会涉及底层的架构设计,但是往往还是需要写一些代码的,所以很排斥写代码的同学也许不是那么适合这个方向(虽然现在似乎不管什么方向都需要会写一些代码了,悲)。大家可以通过大一的两门公共程序设计课程感受自己对写代码的接受程度。工算的课程比较广比较杂,所以单论数理基础肯定不如理力方向打的扎实,课程难度也相对低于理力,如果对于力学强烈感兴趣或者对知识体系有很高的要求,建议去理力方向(当然这个要求是其他所有方向都满足不了的,不只是工算,工算大概已经是理力之外对数理能力要求最高的方向了)。不过反过来,对于像我这样对力学没太大兴趣,想要尽可能规避力学课程,同时又比较喜欢程序设计的同学,工算就是一个很不错的去处。一些有志转码但是大一的时候还没下定决心转专业的同学也可以选工算作为一个跳板。

最后,关于工算方向未来能做的事情,我想说,事在人为。我觉得工算的同学在数学、力学和编程上都能够打下良好的基础。课程太杂的弊端常会让我苦恼本科好像没有系统地学到什么,无法直接将课堂所学应用于实践当中,但是不可否认的是,在遇到一些需要学新东西的场景时,这种通才的本领能够帮助我们比较快地上手。以保研为例,工算的同学可以回头继续研究力学(其中工程力学系应该是和工算方向直接对口的),可以去对数学和编程有较高要求的工管系、控制系和机器人系,也可以去信科的几个研究生院寻求机会,甚至可以找一些与本专业无关但是恰好需要会做计算的同学参与研究的实验室。总体而言,工算的选择还是非常多的,重要的是敢不敢去尝试,保研的这几种途径里,大多数都需要主动地联系导师争取机会。尤其是跨院系保研,很多同学包括我自己都对自己的能力不够自信或者觉得专业背景不够契合而畏手畏脚,但是也有一些同学去尝试了就获得了offer。希望大家可以趁还有时间多去尝试,定好目标大胆去冲,我相信我们专业的出路基本都是挺好的。
\subsubsection{(2)2022级\ 曹林博}

我将从专业感悟、课程内容和学习心得三方面为大家介绍。

首先,从专业研究内容来说,工算需要通过计算机解决工程中的问题,比如设计、优化、仿真预测等问题。这里我引用袁子峰老师(力学与工程科学系助理教授)的观点:“我们在解决工程问题时,利用仿真方法得到的数值结果是对客观现象的近似描述,而这个过程是多个层面的问题。”我们需要经过理论公式的推导,构建合适的数学和力学模型来描述解决的问题中折射的客观现象(通常是偏微分方程体系),其次需要通过设计与证明(如稳定性、收敛性、鲁棒性)来生成一个可行性好的算法,最后通过代码编程实现这个算法。当然,工算专业课程的广泛性与紧凑性,也为我们研究其他领域内容提供了可能:比如代码能力的培养让我们将人工智能融入研究内容中,数学物理思维的锻炼让我们也能够投身北大老力学的研究,甚至解决超越力学的物理过程的科学计算问题。因此,我们可以在本科生阶段(主要是前三年)通过旁听导师组会、与学长学姐交流、与老师面对面交谈、参访实验室等方式,探索自己的研究兴趣点,以为研究生阶段做准备。

其次,从课程体系上说,根据我们的研究内容,我们需要上数学类、力学类、计算机类三大类课程。

力学专业是非数学专业里面用数学用的最勤的,对数学的应用可以达到“润物细无声”的地步,因此我们需要扎实好数学基础,这也为以后对于工程问题的符号化表述和逻辑性解决提供数学基础。工学院的数学课程比如常微分方程、工程数学、数学分析系列、高等代数等,需要将老师的讲义、ppt、参考教材结合起来,达到相互补充帮助理解的作用。

力学类课程是我们工学院的看家本领,是我们区别于纯IT人员的重要特质。掌握力学的知识体系,才能为我们设计算法提供科学基础。理论力学、材料力学、流体力学等力学课程,需要我们通过习题的训练,将力学的知识应用在具体问题中,从而提高对力学知识的掌握。

计算机类课程培养我们使用计算机的能力,旨在将计算机作为一种工具,从而帮助我们实现模型的建立和问题的优化。计算概论、数据结构与算法会让大家学到编程语言最基础的语法和数据结构,计算几何、计算机图形学、并行程序设计等课程会在提高编程能力的同时,拓宽大家的视野,亲身体会到计算机是如何为科学研究发挥作用的,初步了解工业计算软件的代码编写规范,为我们提供了解科研和投入科研的机会。

最后,从我两年的学习心得出发,想对我大一时候存在的疑惑进行解答,这些问题可能大家初入工院的时候都会遇到。首先大家需要摆脱高中时候凭他律学习的学习方式,大学里的时间和精力是自己自由分配的,因此需要大家对自己的学习情况和生活情况做到心中有数,学会把握自己的生活节奏。其次大家需要多与老师和学长学姐交流,放下羞涩。老师和学长学姐们都是非常热心和善良,乐于帮助解决大家问题的,这也有利于培养大家与他人相处的能力。最后大家要对自己有信心,相信自己,追求热爱,在北大绽放自己的光芒!

祝大家能够在北京大学工学院这个大集体中,熠熠生辉,光芒万丈!


\subsection{能源与环境系统工程}
\subsubsection{(1)2022级\ 付杨}
能源与资源工程系成立于2005年,并于2006年开设能源与资源工程专业,2012年更名为能源与动力工程专业。随着化石能源的过度开采与生态环境的日益恶化,能源学科与环境学科之间的割裂也逐渐显现。2016年,专业更名为能源与环境系统工程,强调能源—资源—环境的一体化。

学习内容方面,大体可以分公共基础课、通选课和专业课三方面谈,其中前两方面全校情况类似。下面主要介绍一下能源专业的专业课。专业课分为专业基础课、专业核心课、专业选修课等类别。专业基础课主要是数分、线代、普物、普化等基础数理化课程,这些课程工学院大部分专业都需要学习。其中,数分是大家进入大学遇到的第一门数学课,理解和掌握难度都较高,且不同老师的考核难度也差别较大。普物分为I(力学、电磁学)、II(热学、光学、近代物理)两门课程,是在高中物理基础上进行的延伸,因为课程内容较多,所以可能需要在课后花较多的时间。普化课程内容个人认为绝大多数为高中化学内容,学习难度不大,且老师给分较好。

专业基础课是全院一起上的大课,而专业核心课和专业选修课则大多是能源专业的小班课。每一门小班课都向我们展示了能源专业的一个方面,或让我们掌握能源专业的一项能力。值得一提的是能源与环境工程导论和本科生实践大课堂两门课程。能源与环境工程导论是能源专业的第一门专业课,它告诉我们能源与环境二者之间千丝万缕的联系,也借由各位老师的研究方向让我们看到能源专业的广阔前景和发展前沿。本科生实践大课堂是能源专业本科生实践的特有形式,考核方式为开题、中期、结题三次答辩。在整个周期中,我们可以每周参加老师的组会,定期和老师讨论课题进度并向老师寻求建议,从而对科研工作有一个初步的了解和认识。与一般的本科生科研相比,本科生实践大课堂周期较短,且所有选课同学统一答辩,能有效调动大家科研的积极性。

另外,理论与应用力学(能源与环境系统工程方向)(即强基能源方向)与一般的能源专业相比多了几门力学类的专业课,如理论力学、材料力学等。相应地,一些能源专业的必修课也变为选修。以上是对能源与环境系统工程专业学习内容的简要介绍。

下面准备简要谈谈北大能源专业的未来发展方向的问题(即科研和就业)。为实现“碳达峰、碳中和”的目标,很自然的有两个发展方向:一是回收现有CO$_2$,二是减排未来CO$_2$。这也对应着能源专业科研的两大趋势:CCUS(carbon capture utilization \& storage,即碳捕集、利用和封存)技术和新能源。CCUS技术尝试从大气中直接或间接收集CO$_2$,并将其进行二次利用或封存在地下。而新能源则可能会与材料等学科合作,聚焦于新型电池等领域。除了这两个方向外,也有部分老师聚焦于界面及相变、极端条件下的传热、热技术管理等课题。总体看来,北大能源专业教员规模虽然不大,但每位老师的研究方向重合度都不高,能覆盖大多主流的能源前沿方向。且除保研本系外,大家也可以去隶属于地空的能源研究院或材料学院等与能源联系密切的院系深造。当然,能源系老师们的背景国际化程度都很高,对大家出国留学也是有利的。

本科就业的情况较少,且竞争优势不大。如果有就业计划,可能要考虑在本科阶段去相关企业做几段实习。能源专业在社会上对口方向较多,如比亚迪等新能源车企、国家电网、中石油、中石化等。当然,也有同学从事对口性没那么高的职业,如互联网大厂等。不论如何,北大的文凭和大家的学习能力是大家有力的敲门砖。

最后写一点在能源系学习生活的体会吧。能源系的课程难度没有力学系那么高,同时小班课的模式和乐意与同学们交流的老师也保证了大家的学习效率。所以只要大家认真听讲并扎实完成课后的各个环节,就能取得一个不错的成绩。同时,能源是与现实生活紧密相关的一门学科,能源系的老师们也经常在课程设计中引导大家在真实场景中思考问题并尝试运用所学解决问题。比如,《工程热力学》期中大作业是为北京市设计一个零碳供暖系统并在一个3000平方米的民用建筑上运用。在能源系学习总能有一种使命感,因为你知道自己所学的专业能为人类更美好的未来作出贡献。能源系的师生氛围轻松融洽,且学生人数较少,所以大家都能有机会和老师充分交流。欢迎大家选择能源与环境系统工程专业,欢迎大家加入能源系!

\subsubsection{(2)一位希望保持匿名的同学}
\textbf{一、专业情况概述}

\begin{figure}[htbp]
    \centering
    \includegraphics[width=0.8\linewidth]{1.3.3徐克老师问答.png}
    \renewcommand{\figurename}{图}
    \caption{徐克老师知乎回答}
    \label{fig:enter-label}
\end{figure}

转自能源系徐克老师知乎回答,恰如陈正老师在树洞5G冲浪一样,能源系的徐克老师活跃在知乎一线。徐克老师对保研、就业、老师科研方向、课程设置等等都进行了较为详细的讲述,同学们可以重点关注第3点。

\textbf{来源:}\href{https://www.zhihu.com/question/583497117}{\textbf{https://www.zhihu.com/question/583497117}}

\textbf{二、能源专业学习内容}

1.非工学院(体育、通识、政治、英语)课程。对于能源系的同学们来说,由于本专业课程难度相较理力等专业课程难度低,同学们可以有更多时间学习感兴趣的通识类课程,建议提前在树洞、课程测评等网站善用搜索,尽量不要选择给分迷惑、努力小于回报的课程即可。

2.基础必修课

\begin{itemize}
    \item 数理基础课程:数学分析一、二(非强基同学可用高等数学B上下替换)、线性代数与几何、概率与数理统计、普通物理ⅠⅡ、普通化学B、(常微分方程)、(工程数学)……

    \item 力学类课程:工程流体力学、(理论力学B)、(材料力学B)、……

    \item 计算机基础课程:计算概论、数据结构与算法、计算方法……

\end{itemize}

括号标注的课程对于工科试验班(非强基)的同学是建议的自主选修课程,对于强基的同学是专业核心课(必修)。

大部分基础课程都会在前三个学期完成,对于能源专业来说,在培养方案上减少了部分数学课程,部分课程难度相较理力等专业降低,如用工程流体力学替换了流体力学。因为强基同学必须以理论力学为主修专业,所以理力专业部分核心课程仍是必不可少的。

这里需要注意的是开设在大一下学期的普通化学B和高等代数两门课程,由于在第二学期结束时才会选择专业方向,而部分专业要求必修普化B,部分专业要求必修高代,可能会需要在大二、大三之后补选相关课程。对于确定选择能源专业的同学来说,在大一时可以选择不修难度较大的高等代数。

虽然常微分方程、工程数学不是能源专业同学必修课,但在后续的学习中,相关课程知识还是会被用到,比如在工程流体力学中就会涉及求解常微分方程、以及《工程数学》中的复函数等。

3.专业核心课
\begin{itemize}
    \item 热力学课程:工程热力学、传热传质学、热统导论……

    \item 能源传输转化类:传热传质学、热值输运模拟……

\end{itemize}


能源系专业课相对难度较低,易于理解,大多是基于热学、力学之上的课程,部分涉及到化学中的热力学的部分,如吉布斯自由能、熵、焓等相关概念。有努力就会有回报。

4.能源专业特色课程

这部分主要列举一些个人认为能源系在培养方案中比较有独特性的专业课。

\begin{itemize}
    \item 《能源与环境系统工程导论》(能环导),这门课是能源专业大二上学期开设的专业基础课程,类似于工学院在大一上学期开设的《工学通论》,2023年秋季由邓航老师开设。课程前半学期主要由邓航老师讲解环境保护与修复、后半学期由能源系多个老师以讲座的形式讲解能源和资源的开发、生产、应用,老师大多会结合自己目前研究的方向课题进行介绍,是了解老师,选择自己感兴趣方向的机会。

    \item 《金工实习》,这门课程是由材料学院开设于暑期的能源专业必修课程。会安排去昌平校区住5天左右,住宿环境很好,课程给分很好,涉及到车床加工、3D打印等等多项具体的加工技术操作,是一门很好的工程生产实践体验课。

    \item 《本科生实践大课堂》,这门课是能源系面向大三年级开设的科研实践类课程,可以说是能源系自己的本研。在第一门课会有多位老师介绍自己的项目供同学们选择,除了能源系自己的老师外,还有工管、先机、材料专业做与能源方向有关的老师。学期中主要跟随相应老师进行科研实践,期末以答辩汇报的形式完成。

\end{itemize}



\textbf{三、专业未来}

大部分同学主要的选择都是升学科研,能源系提供了很好的本科生科研环境,有利于同学们在本科阶段找到自己希望在研究生阶段继续学习研究的方向。此外,能源系大部分老师都有海外科研背景,国际化水平高,对于希望申请出国留学的同学来说也有一定的帮助。

\textbf{四、学习体会}

1.培养方案注重理论基础,学习过程注重工程实践。能源专业是一门工程类专业,但在工学院整体注重数理基础、能源技术更新速度快的影响下,能源系的培养方案切割了传统的工程技术、更加注重理论类课程,而工程实践类的技能则是在鼓励同学们参与到老师的科研课题中学习的。

2.以兴趣为导向的自我探索,非常nice的师生氛围。能源专业课内学习相对不卷,但可能随着专业人数逐渐增加有所变化,同学们可以将更多精力放在本科生科研上或其他自己更感兴趣的领域。能源系经常组织一些师生聚餐、学长学姐本研升学经验分享等交流活动,帮助大家多次、详细地了解老师们的研究课题,找到自己感兴趣的方向。除此之外,每学期还会有班主任、辅导员交流聊天环节、期末考试学生宿舍走访慰问等等,及时解决同学们学习生活中的困难。


\subsection{航空航天工程}
\subsubsection{(1)2020级\ 王一诺}
学弟学妹们,你们好呀!欢迎你们加入北京大学工学院这个大家庭!接下来由我向大家介绍航空航天工程这个专业吧。

我们学院航空航天工程系成立于2010年5月,距今已有十四年,因此课程安排、培养方案等都设置的较为合理。我们航空航天的课程,侧重数理基础,除了大一时必修的数学分析、普通物理、线性代数与几何,后续还有概率与数理统计、工程数学、理论力学、高等动力学(我更愿称之为理论力学下)、工程流体力学等。此外,作为航空航天的学生,还需要上很多航空航天相关的课程,比如空气动力学、飞行器结构力学、飞行器设计与动力以及航空航天概论等。另外,系里还开设了电路与电子学、控制理论基础、工程热力学、航空航天信息工程等课程,这些课程对应着我们航空航天系里老师们的研究方向,比如流体、声学、燃烧以及空天信息等。

由于存在大量数理课程,我自认为航空航天的课程是学院里和理论与应用力学系最为相似的,但我们课程的难度比理论与应用力学系要小。不容忽视的是,大量的数理课程使得航空航天的课程学习依旧存在着一定的挑战。

这几年我们系大多学生都选择了本院升学,我也不例外。所以,对于想要在本院(或者说本系)升学的同学们,我有一些小建议。首先,如果你想了解某位老师的研究方向和主要研究内容,不能只看学院个人网页上的研究方向介绍,而应该阅读这位老师曾经发表过的论文,因为很多研究实际内容和你所认为的有较大出入。其次,建议在大二下或者大三找老师做本研,提前上手体验科研,这能够让你更直观了解你所感兴趣的研究方向。当然,体验过后也可能发现这个领域的研究并不是你所想的那样,因此,本研也相当于一个试错的机会。

这几年我们航空航天出国的同学比较少,但我认为我们系还是可以出国的,并不会因为“专业限制”而受到很大阻碍,如果是强基的学生,书面上我的专业是理论与应用力学系,受到的影响应该会更小,所以我认为如果有想要出国的同学也不必太过担心。

最后,欢迎大家选择工学院航空航天工程系!

\subsubsection{(2)一位希望保持匿名的同学}
大一的课程安排各个方向大致一样。主要是诸如数分、线代、普物的数理基础课和计概数算这样的计算机基础课。在大一下学期,航空不要求修高代,需要修普化。但是,对于想做控制方向相关科研的同学也需要了解一下相关知识(当然不一定要选修或者在大一下选修这门课,可以自行了解,或是放在比较空闲的学期选修)。

大二开始,各个方向的课程会出现一定的差异。航空相对于理力的数学要求会降低,这既是一件好事也是一件坏事。相对来说数学课的课业压力相对降低,但是数理基础会存在一些不足之处。例如,大二下的极简版数学物理方法即工程数学教授的知识,对于将来想做声学方向研究的同学确实不足。建议还是需要自学完整的数理方法。航空的力学课部分要求同理力一样,另一部分要求相对较低。(由于本人学过物理竞赛,这一部分课程难度感知不是很准确,所以不做过多阐述,总之是小马过河)。此外是航空的特色课程,如电路与电子学,工程热力学等。这些课一方面偏向于应用,更具航空特色,另一方面这些课也是了解导师研究方向的好机会(毕竟大多是航空自己的老师开的)。

当然官方的课程安排是一个范例,各个同学在研究方向上不同,所以这个范例在细节上并不适合所有同学(而且有些课也许不能正常开设)。在具体的操作层面上在保证不撞课(参考往年该门课的开课时间,咨询教务)的情况下可以微调选课顺序,例如对机器学习感兴趣的同学将概率与数理统计提前放在大二上修完全没有任何问题,基础物理实验如果不和空气动力学基础撞课也可以放在大三下修。(顺便提一句,航空航天信息工程这门课目前开不了,本校同学也很少会选择走空天信息方向)

前文反复提到研究方向,至于怎么选择研究方向,建议如下:多多阅读感兴趣的老师曾经发表过的论文,因为很多研究实际内容和你所认为的有较大出入;阅读论文也会发现自己知识欠缺的地方,查找对应的课程教材进行翻阅,进一步判断自己适不适合这一方向。

\subsubsection{(3)2022级\ 郭祺}
航空航天工程系(以下简称航系)每年分流后的人数在18左右,像22级只有14人,从这点也可以看出其实航系的学习压力没有那么大,有些课程是和元培空飞班(如飞行器设计与动力和飞行器结构力学)一起上的,有些课程可以去上能源方向的,一些课程难度也会比较低,给分比较好,但是在北大给分和学到东西一般来讲不可得兼,建议同学们权衡一下。由于我是非强基学生,选课的弹性比较低,基本就是按照培养方案来选,除部分课程能源与航系学分互认,剩下的都是正常按方案走就好。 

工学院每个学期都会开一些高级课程或研本课,比如多相流、燃烧理论与模拟,气体力学、振动理论等。但这些对读研生涯有益的课程未必每学期都开,如果遇到并且确信自己可以坚持下来的话,还是建议一选的。航系的不少课程都会有大作业,建议在分流之后迅速结识同专业的朋友然后组队。本人是从物院平转进入工院的,可以明显感觉到两院之间的氛围不同,物院人大多比较外向积极,讨论各种问题,工院人(我在的集体里)一般比较分散,所以为了在课后时间投入最多的工学院生存下去,大家很有必要积极地向外社交,交换学习资料和心得,相互打气(本人大二真实写照,没有那些同学真的会压力拉满)。

我多说一些关于科研方面的事情吧。目前来讲,北大在航空航天上还是更偏向“航空”,同时绝大多数老师都是流体力学方向的,比如声学、气动、湍流、燃烧等。如果大家更喜欢“航天”,如火箭等,可以看看老师近期的文章或是与老师作进一步讨论;如果大家更喜欢卫星(毕竟也是航天),那么地球与空间科学学院可能是个好的选择;如果大家更喜欢行星科学,同样地,地空是专业的,物院的大气与海洋科学系也有老师做这个方向。光看老师的个人主页介绍可能比较蒙圈,所以建议大家积极与老师联系,深入了解相关领域,看看是否与自己的兴趣相符。此外,部分力学系的老师也从事航空航天方面的研究,如果航系老师们的方向没有自己心仪的,不妨去翻一翻力学系老师们的介绍。如果大家想从事流体力学方面的研究,特别是湍流的研究(这是北大的门面),可以多点了解力学系老师的研究进展。

由于历史原因,工学院的研究风格更偏向理论,这种风格也影响了航系的发展,所以教授们更多是从事理论研究或者计算物理学(如计算流体力学),也就是要求较高的数理能力与编程能力。当然,工程不能脱离实验与实践,如果同学们对做实验更感兴趣,不妨在与老师面谈的时候明确表达自己的意愿,随着工院新大楼的进一步建设与更多实验室的落成,需要的实验人才也会越来越多。

\subsection{生物医学工程}
\subsubsection{(1)2022级\ 范文琳}
生物医学工程是一个包含内容非常广泛的专业,几乎任何涉及生物或医学问题的工程类实践都可以被划进这个范畴。因此,相比于具体知识的传授,生医系的专业课更加注重项目式的学习,并且会对一些科研和工作上的“软”技能有一定的侧重(比如小组合作、presentation、文书撰写等等)。生医系比较有特色的课程是系列PBL课程(i.e.生物医学工程原理、生物医学工程设计),在这些课程中老师会要求同学们以小组为单位在一个学期内完成一个课题的文献研究或者实现一个小型项目。过程中同学们可以向任何可以得到的资源寻求帮助,主管课程的李长辉老师也会为各组提供一些如何求助的指导与建议。
科研方面,生医系对于本科生提早进行科研实践给予极大的鼓励。生物医学工程专业对口的老师很多,但具体的研究方向之间也相去甚远,包括成像、计算、电子,也有偏向纯生物的课题组。大家可以在未来技术学院官网上查看老师们的研究方向,通过和老师约谈或者听组会的方式来决定自己更加感兴趣的方向。一般而言,大家会在确定自己的方向之后在选课上进行一些调整,以规避交叉学科学习中通常存在的“博而不精”的问题。
另外值得注意的是,尽管生医系的强基生被要求修一系列力学系课程,但这些课程在难度要求上和正统的力学系课程还是有一定的区分。当然,任何理工类学科中数理基础总是重要的,诸君可以自己多多留意(x)。

\subsubsection{(2)2022级\ 李昆泰}
\textbf{生物医学工程系概况:}

生物医学工程系(常简称生医系,后文亦沿用,这个名字可能小有误导性,之后会在文章中详述)和材料系是工学院“系”、“专业”或“方向”中较为特别的两个,因为实际上生医系和材料系的“研究生院”都是单独成院的,它们对应的学院分别是未来技术学院和材料学院。未来技术学院这个学院可能在大家看来相当“科幻”,觉得我们搞的好像是什么天顶星科技一样,在某种意义上确实,我个人也觉得我们在创造一些极为有意义的工程发明和科学发现。更实际地说,北大的未来技术学院面向的就是未来的生命健康,所以也无怪乎生物医学工程系是在未来技术学院中的一个系了。简单理解的话,生医系是未来技术学院的一个系,而这个系的本科生放在了工学院这个学院中进行培养。大家如果选择国内深造,大概率可能会选择进入到未来技术学院的课题组中。

\textbf{生医系的入学渠道和同学构成:}

生物医学工程系现在应该是有三个入学渠道。一个是高考裸分录取中的“工科试验班”类,一个是强基计划中力学类的“理论与应用力学”强基的双学位,还有一个是强基计划生物类的“生物四”。其实在大一选专业后实际进入生医系就读时没什么差别(生物IV我现在不清楚),编班都在生医班,绩点排名、评奖评优什么的都在一起排,也享有同等的进实验室等等的资源。差别主要在培养方案上,强基需要比非强基的同学多修一些力学类课程,数学类课程的难度要求可能也会高一些;非强基的同学则需要修读更多学分的生医工专业相关的课程。

\textbf{生医系在做什么:}

前文说了,生医系是未来技术学院的一个系;同时,生医系还是未来技术学院唯的一个系。大家未来路径选择上很大的一个优势就是——本科从生医系毕业,研究生可以选择整个未来技术学院的方向(当然也可以选择如生科、人工智能等其它学院的方向,这里讨论的是主流的选择)。所以,在说是讨论生医系在做什么,其实更多讨论的是未来技术学院整体在做什么,乃至整个生物领域-医学领域-工程领域这个交叉点上大家在做什么。

先从“狭义”的生医系本身讲起。这里要澄清一个长久以来的误解:生医生医,其实是生医工这个“生物医学工程”的简称的简称,“生”和“医”是工作面向的核心问题和应用场景,“工”则是这个专业的“技能”核心。大家可能会认为生医系是做生物科学研究的?还是当医生的?进检验科的?等等等等,其实都不甚准确,或者说不是标准的路径。准确的说,生医系可以说是给生物科学研究造更好的仪器设备,让科学家看得更清楚,操作地更“稳准狠”,打开更多的研究可能性的工作;是给医生创造更好的医疗器械和临床工具,或者给患者本身制造更好的健康检测、干预与辅助设备,增进病人的生存率和福祉,为更多疾病带来创新型的治疗与干预方法的工作。它的学科基础主要是工程学和物理、化学、生物学;关注的是解决生物科研和医疗临床问题(北大生医系前者做的较多,但社会上主要谈的是后者);而在未来,它更有可能带来大幅长寿、人机融合等人类社会的重大变革。

在临床领域,生医工所做的可以看到的工作已经非常丰富了。大到CT、磁共振、超声等各种医疗器械,小到血压计、血糖仪、血氧仪等日用医疗健康产品,再到嵌入到智能手表里的心率检测等功能,都是生医系(与其他学科交叉融合)的创造。脂质体药物、CAR-T等创新疗法其实也可以称之为一种“针对生物的面向医学需求的工程学”,所以这个学科有相当的学科交叉性和知识结构多样性。进一步的,在“广义”的生物-医学-工程领域上说,我们会在各个尺度、各个物理原理上进行创新。我们会利用力、热、光、电、声等物理原理,酶促反应动力学等化学原理和激素调节等生物学原理,一方面探测生物从分子细胞到个体群体的各种信息,另一方面用这些自然的力量去扰动、调节乃至控制生物系统,以增进对生物系统的理解和对人类调控生物系统的手段的掌握;另一方面,我们会从物理的疆域跨越到信息的疆域,利用算法、大数据与人工智能的力量对我们获得的信息与能力进行更强大的总结、提炼与应用。这些抽象的理念具体化到各种方向中,就是生物成像(利用光学获取细胞及亚细胞尺度的信息)、生物力学(工学院可能做的比较多,提取生物学现象中所蕴含或利用的力学规律)、神经电生理和神经工程(通过电信号收集生物体的信息和对生物体进行扰动或控制)、生物传感器与生物电子学(利用电子元器件构建对不同物理信息的提取装置)等等、生物探针(利用蛋白质等生物分子的相互作用原理与光学等物理原理实现对生物系统的观测)等等。

未来将技术学院还有分子医学所、国家生物医学成像科学中心、大数据与医学人工智能系几个机构,其研究方向其实也在我刚刚讨论的广义生物-医学-工程交叉领域之中。相关的研究方向和实验室也是大家可以去尝试、选择的。在真正的科研实践当中,尤其是生物-医学-工程这个大的领域之中,科学与工程、发现与发明、研究与创造、提炼与综合一直是相互交织,相生相荣的关系。我们很难划出一个到底什么是生物科研,什么是生物医学工程,什么是分子医学研究的分界,但只要大家对生物学问题、医学问题和针对二者的工程问题中的其中一点有所兴趣,都欢迎大家来到生物医学工程这个广大而充满无限可能的领域中探索与创造。

        
\subsection{材料科学与工程}
\subsubsection{(1)2021级\ 李一川}
材料专业学习的知识会涉及更多的领域,包括物理、化学、生物等多个方面的知识,课程内容也比较丰富,同时,在注重基础知识学习之上,会更加强调与实际应用和生产生活相结合。课程讲述的内容也非常前沿,许多也是当前社会发展的热点话题,比如钙钛矿半导体、新能源电池、柔性电子等等。材料学院目前已经单独成立学院,导师数量众多,涵盖的研究范围也非常的广,对接资源非常丰富,同学们可以根据自己的兴趣,非常自由的选择导师加入课题组进行本科生科研训练。

材料专业的本科教育更加强调基础知识的学习,以及大量细分领域的了解,拓宽知识面,而大部分同学会选择在本科毕业后继续在某一个专业领域继续深造,攻读硕士或博士。
\subsubsection{(2)2022级\ 曾帅鹏程}
\textbf{关于基础课程:}

1.全校通选:全校学生情况都差不多,但作为工院学生尤其会对三四类通识感兴趣,所以务必注意一二类通识选择时一定要选核心通识课,避免多修。

2.工院基础:以数分,高代等专业课程为代表,在大一未分流时,全体学生情况统一(强基与非强基会略有不同,具体请研究培养方案)值得注意是工学院对数理要求较其他理学院来说有过之而无不及,且范围广(计算机,数学,物理,化学都有涵盖)务必对此多加上心,若对材料专业感兴趣,建议大一下修完大学化学,减少之后学期负担。

3.材料专选:在大一后分流来到材料专业,需要修习许多专业课程,强基生在不能完全丢掉理论力学的学习课程任务量会比大一学年任务量加重。理论力学方向相关课程会和生医、能源共同一起开班上课,难度略低但也不可轻率。材料方向课程则以小班授课形式进行,代表课程有:材料科学基础,材料物理、物理化学、材料学中的量子与统计……这些课程理论性极强,会结合当前材料方向前沿进行讲解,内容在化学基础比花园较弱的情况下会显得晦涩难懂,但很有必要花费大力气进行深度学习。工学院的课程课后花费时间一直位于北大第一,希望同学们早做心理准备,但也要相信,只要肯下功夫,成绩也不会辜负你的努力。

4.自主选修:这方面课程需要和你的研究方向想结合,材料学的知识十分广博,在确定好自己今后学习方向后特异性极大,结合方向多修习校内相关课程,方能使学业一帆风顺。

\textbf{专业未来:}

材料学本科就业不占优势(其实除去金融,很少有专业会愿意本科后就业)在读完研后(听说本校通常给博)就业方向极广,材料学科囊括范围极多,从本校导师研究方向便可管中窥豹:肿瘤疫苗,电催化,锂电池,纤维、钙钛矿、高分子等等,所以需要确定尽早从事材料的科研方向,毕竟任何一个方向都需要精深学习,都需要花费时间。在读完硕(或博)后,有许多出路:博后发展、企业研发、高校任职、科研所任职等。具体内容都是因人而异。

\textbf{学习体会:}

在化学基础较化院薄弱的情况下,每一门材料的课程都会显得晦涩难懂,每一门专业课程都需要在课后投入大量时间进行学习,但要相信,每一分努力都会带来回报,材料专业对科学前沿的了解十分重要,所以了解自己方向的科学前沿发展情形都很有必要,需要自己课后花费时间进行了解。


\subsection{机器人工程}
\subsubsection{(1)2021级\ 刘家豪}
\textbf{专业内容}

从本专业的必修课设置看(详见培养方案),机器人系统的设计、制造中所涉及的各环节内容均有覆盖,课程涵盖集成电路、控制器、机械设计、建模等内容。个人认为通过本科阶段的学习,了解制作完整的机器人所需知识,掌握MATLAB、SoildWorks等常用软件的使用,甚至自己DIY一些简单的机器人应不成问题。在专业必修课的主线之外,亦有较为丰富的选修课程可供体验,譬如机器学习、嵌入式、图论、计算机视觉等相关课程。

总体来看,作为一个多学科交叉的本科专业,机器人工程的课程设置更注重广度,培养能力全面的人才;建议结合本科课程的学习、实验课体验、本研经历选择自己感兴趣的具体研究方向。如果对机器人领域感兴趣的话,相信本专业能够给你良好的学习体验。

\textbf{自身体会}

就我个人三年来的学习体验而言,与其他工科院校的类似专业相比,本专业也较为注重数理基础的培养,这也是本校工学院的院系特色之一,具体体现为大一(选专业前)需要完成多门基础数学课的学习,后续也有较多数学类课程可供选择(选修居多)。学有余力的同学可以尝试本院的“理力+机器人”双主修项目。

若要达到精通某一具体的机器人领域,以作为将来科研或就业道路的立身之本,我认为仅完成本科阶段的课程学习可能并不足够,课外的自学和实践也很重要。本科学习期间适当参与科研项目或竞赛更有助于自身技能的完善,如大二下学期报名参与自己感兴趣的本科生科研项目等。

\textbf{未来去向}

若想继续从事机器人领域,相比与直接本科就业,我个人认为读研深造可能是更为合适的选择。下面我就我个人的了解大致介绍与本专业相关度较高的科研方向。由于本专业涵盖较多学科的内容,相关可供选择的科研方向亦丰富多样,在此仅举一些我较熟悉的例子。例如从事某一细分领域的研究,以机器人感知为例,硬件如视觉、触觉传感器等的设计,软件如多传感器融合算法、SLAM等,都是可选的方向;又如完整机器人系统的设计,譬如特定用途的机器人如某一类医用机器人、水下机器人、软体机器人等;将AI与机器人融合也是时下较为热门的方向。另外,进入工学院官网浏览本系老师的研究方向或咨询学长学姐等,或许能让你对本专业的科研方向有一个较为全面的认知。

我个人对于本科就业了解不多,在此也就不过多介绍。进入相关科研院所工作、加入科技公司或某些大厂的机器人研发部门,一般是较为对口的选择。当然,有了北大计算机通识课、专业相关课程所积累的编程基础,进入互联网企业工作也不失为一个选择。倾向于就业的同学可以多多留意院系宣讲时介绍的毕业去向统计或咨询学工办老师。

\subsubsection{(2)一位希望保持匿名的同学}
机器人工程专业涉及多个领域的基础课程,包括力学(理论力学、高等动力学)、数学(高等代数、常微分方程)、电路(如数电、模电)、机械(如机械设计基础、机器人学概论)、自动控制原理等。这些课程是入门的基本要求,也是未来深入学习和研究应用更具体内容的基础,需要大家都能有一定程度的掌握。因此,对于想要选择机器人专业的同学来说,前两年的首要任务就是重视这些基础课程的学习。

除了基础课程的学习,第二个关键任务是主动探索本专业的研究方向。机器人专业涵盖了多个研究方向,包括智能系统与控制、群体博弈与智能决策、医用机器人、水下机器人与装备、先进制造与工业软件、无人系统的自主决策与规划等(大家可以到工学院官网查看更具体的内容)。建议大家积极利用各种机会,如现代工学通论课程、各类讲座,以及与老师的交流,去更加深入地了解每个方向的研究内容。在还没有明确的兴趣时,多与老师交流将大有裨益。在对各个研究方向有了一定了解后,就可以选择一个合适的时机,主动联系老师,参与一些本研任务。最初可能有忐忑和顾虑,但是这种边学边用的学习模式,给予我们很大的成长进步的空间,不仅能够加深对知识的理解,还能提升解决实际问题的能力,最重要的是也能检验我们是否真正感兴趣(如果发现自己实在不感兴趣,那么就去探索下一个方向,试错空间很大的!)这对我们的自身发展能起到更全面的帮助。

此外,在后续的课程学习和本研中,MATLAB、Python以及SolidWorks等软件将会频繁使用。因此,建议大家提前自主学习+练习这些工具的基本操作内容(迟早都需要自学的;在学习计概、数算的基础课程时也可以考虑选择Python)。

希望大家在选专业这方面上不会有太多的困惑啦,如果有,那就及时主动寻求帮助!愿大家能够享受这段美好的时光!

\subsection{环境科学与工程}

\newpage









\chapter{选课指导}
\section{课程分类}
\subsection{北京大学课程分类}
详情见开学期间会发放的《北京大学本科生选课手册》和《工学院本科教学手册》,这两份文件极其重要,请务必仔细研究。

毕业之前,必须修完的学分(见以上两份文件)约为140学分(也就是说,修完这些学分和对应课程是你拿毕业证书的必然要求,但并不代表你只能选这些课,没有人拦着你修更多学分来完成自我实现)。

\begin{figure}[htbp]
    \centering
    \includegraphics[width=1.02\linewidth]{大表1.png}
\end{figure}

%%%%%%%%%%%%%%%%%%%%%%%%%%这里有个大表格
\begin{figure}[htbp]
    \centering
    \includegraphics[width=1.02\linewidth]{大表2.png}
\end{figure}



总结:
\begin{figure}[htbp]
    \centering
    \includegraphics[width=0.2\linewidth]{2.1.1衡量.png}
\end{figure}

\section{毕业要求}
1. 毕业要求:

\begin{itemize}
    \item 修完培养方案中要求的学分(或经过教务部审核认定可以转化的学分)
\end{itemize}

2. 推免(保研)要求:

\begin{itemize}
    \item 原则上在大三前修完专业必修课、大部分要求的专业选修课
    \item 强基和非强基同学要求有所不同,详见1.2.2节
\end{itemize}

3. 荣誉学位(额外)要求:

\begin{itemize}
    \item 成绩排名专业前40\%以内(这是新生暂时能做的事情:学好基础课程)
    \item 选修至少5门荣誉课程,其中至少4门课程成绩>=85分(荣誉课程暂时不是新生要考虑的事情)
    \item 获得周培源力学竞赛二等奖相当于1门荣誉课程优秀;一等、特等奖相当于2门荣誉课程优秀
    \item 必须选修本科生科研(新生可以慢慢接触了解本研在做什么、自己的兴趣点在哪里,一般大二大三才会正式进组)
\end{itemize}

\newpage

\section{选课网基本操作}
从“门户-选课”进入选课网。

下面将以时间为主要线索对北京大学选课系统(海淀大赌场doge)进行极其简要的说明。

选课总体上分为五个阶段,分别是\textbf{\textbf{预选、补退选第一阶段、补退选第二阶段、补退选第三阶段、补选}}。又可以简单地根据预选抽签前后把时间划分为两个阶段,前一个大阶段(预选)是主要\textbf{\textbf{选课}}阶段,后一个大阶段(补退选和补选)是主要\textbf{\textbf{抢课}}阶段。

整体时间轴(具体日期每年不同,需要关注教务通知,尤其是暑假学期选课)以2024年秋季学期为例:
\begin{figure}[htbp]
    \centering
    \includegraphics[width=1\linewidth]{2.2选课时间.jpeg}
    \renewcommand{\figurename}{图}
    \caption{2024年秋季学期选课时间轴}
    \label{fig:enter-label}
\end{figure}

1. 预选

(1)把课程从选课列表(货架)添加到选课计划(购物车)

\begin{itemize}
    \item  教务会将一些重要课程提前加入选课计划
    \item 课程加入选课计划后还没有成功预选
    \item 点击课程名称,可以获悉课程简介、所用课本、课程大纲等信息。不过很多信息并不完全准确(有的甚至是十几年之前写的版本),具体内容务必咨询开课老师、学长学姐们。
\end{itemize}

(2) 把选课计划(购物车)中的课程添加到预选单(待“支付”)中

(3) 为预选课程投点(详见2.5.1节)

(4) 等抽签

2. 补退选第一阶段(从此阶段开始,选课操作需要手动填写验证码以防作弊,每次选课or刷新后均需要填写)

(1) 如果有人退课,会让出若干名额

(2) 把课程加入补退选单中(操作同预选)

(3) 等抽签抢空出的名额(几率不大,常常是几十个人抢几个名额)

3. 补退选第二阶段

(1) 抢课最关键的阶段,\textbf{\textbf{先到先得,不抽签}}

(2) 蹲点等别人退课,选课网站刷新出空余名额(退课和名额刷新不同步)

(3) 抢!(操作同预选)

4. 补退选第三阶段

(1) 跨院系选课名额开放,可以选一些其他院系的课程,对于有志于转专业的同学来说比较重要。

(2) 抢课规则同3

5. 补选

(1) 该阶段选中后不能退选

(2) 可以求开课老师收留或者求教务拓名额,比较灵活

6. 投点技巧

此处介绍选课投点技巧。但是并不很重要。

我们每个人一共有99点意愿值,需要在不同课程之间分配。对一个课程来说,其他条件都相同的同学,投点越多选中课程的概率越大。那么我们该怎样分配点数呢?这时我们需要考虑两个问题:\textbf{\textbf{什么课程需要投点;投点的课程要投多少点}}。

我们知道,对于专业课,比如大家一定要学的数学分析(一),开放名额>>学院人数,且无论如何教务会保证同学们能够选上本专业的专业课。那么,这门课完全没有投点的必要,事实上系统中也投不了点(被设置为系统推荐课程)。什么课需要投点?答案是:\textbf{\textbf{供不应求的}}体育、通识、英语、政治等。

我们又知道,新生一学期最多可以选25学分的课程,而专业课占了很大一部分。这就导致了一个问题:比如同学想选多门备选的通识课,但是发现超学分了,通识课塞不到预选名额中。这时可以这样操作:预选阶段不选必中的专业课,拿专业课的学分空档赌多门通识课等,等选上心仪的课后把其他备选课退掉,再在补退选阶段把专业课加回来。这是一种小技巧,更多技巧建议自行体悟。

接下来我们讨论投点多少的问题,对这个问题学长学姐说法不一。

比较热门的投点办法有两个:质数投点法(玄学)和all-in投点法。前者认为只要所有投点都是质数,那么最终选课成功率大大上升;后者不计代价地把大量点数(甚至所有点数)砸到一门课中,以期待一门课的高概率。后一种方法未必不好,因为这起码保证了一门好课大概率选中(当然,也有可能all-in但是没中orz)。如果我们采取平均投点的办法,每门课选中概率都不大,有可能出现所有课全掉的情况(但也有可能全中)。最终的投点办法,还需自行衡量。有一个有意思但未必有用的投点辅助程序可以玩一下。\href{https://wyjjmzx.github.io/pku/toudian.html}{https://wyjjmzx.github.io/pku/toudian.html}


\section{选课辅助工具}
我们常用以下途径判断一个课程的综合好评率:

1. 看“预选名额/计划名额”。直接应用别人的劳动成果,别人搜索完资料后觉得好,因此选课人数很多,那大概率这门课的综合口碑很不错。

2. 树洞treehole.pku.edu.cn,搜索课程简写或者老师姓名简写即可查到学长学姐的课程测评。考虑到树洞的匿名性质,需要同学们注意分辨。

3. 非官方课程测评网站。在浏览器中直接搜索“北大课程测评”,第一个网站即为课程测评网站,该网站完全非官方,且完全不控评,同样地需要同学们注意分辨。

4. 广泛咨询靠谱的学长学姐,获得较为详细的听课感受等信息。

\textbf{核心要义:不仅仅追求给分的良好,更要追求自身兴趣能力的发展。}

\section{特殊选课模式}
\subsection{跨院系选课}
有志于转专业、辅修学位的同学需要注意,别院系的专业课只可以在补退选第三阶段进行选择,需要提前为想选的课预留出空间。

如果没有转专业、辅修学位打算,也可以在这一环节选择别院系感兴趣的课程去学。学分可计入毕业学分中的“自主选修课”。

\subsection{课程替代}
培养方案是灵活机动的。根据培养方案,一些课程之间可以互相替代,进行毕业学分认定。培养方案上没明确写的内容请自行咨询工学院教务。

\subsection{冲突选课和手工选课}
申请会很麻烦,而且一般用不着。如果实在有需要,可以阅读教务部网站-学生-选课和退课。

\subsection{双学位选课与超学分选课}
分为主修+辅双两个选课通道。辅双通道可以进行双学位课程的选择,并且无需投点。通过辅双通道选择的双学位课程需要缴纳相应学费(见手册4.1节)

理论上单学期选课量为14$\sim$25学分。当总GPA$>$3.7时,学生可以申请超学分选课,一般至多为单学期30学分(需要经历一系列复杂的申请、审批过程,见教务部相关网页)。

特别地,双学位学生选课单学期学分上限为30。

\subsection{中期退课}
每学期中期(第八周左右)可以根据教务通知进行中期退课(体育课除外)。退课后,成绩单上该课程记为W,不参与绩点计算。

中期退课是实在学不下去时的应急策略,过多退课可能会让老师对学生的能力产生质疑,可以使用,但不建议过度依赖。

\subsection{不按照培养方案选课}
理论上,同学们可以有自己的选课规划。但是培养方案的设置有很大的合理性,适合绝大多数人的学习进程。如果想按照自己的节奏安排课程,建议与老师、学长学姐深入探讨一下这种方案的合理性。

\chapter{学习生活}
\section{工学院培养方案}
探索过工学院官网的同学可能会发现:工学院近几年的培养方案几乎是一年一改,辅修双学位的政策也时有变动。笔者认为这与2019年开设机器人工程专业、2020年录取强基计划的学生、近年来学院课程开设情况、同学诉求等等诸多因素均有密切关系。



\begin{figure}[htbp]
    \centering
    \includegraphics[width=1\linewidth]{3.1培养方案分列.png}
    \renewcommand{\figurename}{图}
    \caption{2024版培养方案专业情况}
    \label{fig:enter-label}
\end{figure}

下面我们将2024级培养方案与2022级相对比,给出培养方案的一些变化:

1. 主修类方案中增加了3个专业方向。

\begin{itemize}
    \item 理论与应用力学(机器人工程方向):对强基生,相比2022版方案增加了只选择机器人方向的自由度。(2022级方案只能选择理力+机器人双主修轨道)

    \item 生物科学(生物医学工程方向):强基生物科学类IV

    \item 化学(材料科学与工程方向):强基化学类IV

\end{itemize}




2. 缩减学分:相比2022级单专业144$\sim$150学分,双主修160$\sim$166学分;2024级方案缩减到单专业134$\sim$142学分。双主修151$\sim$157学分。

3. 强基方向不再必修“基础物理实验”课程。

北京大学在理科方面的强势使其在发展新工科的过程中选择了“以科学促工程”的思路,这一思路也直接反应在了工学院的培养方案上:重视理论教学,而实践、设计、研发、创造类的培养较少。这会给工学院同学们未来的科研道路和就业带来潜移默化的影响,而每位同学需要了解的是:学院能为我们提供什么、无法为我们提供什么。学院暂时无法顾及到的那些方面,同学们可以有意识地自己寻找资源进行尝试、探索。

回到培养方案本身,工院培养方案给予同学最大的优势就是扎实的理论基础。纵观培养方案,偏理论的专业课占据了绝大多数学分;并且学习过程中辅以不少练习和考试,认真学下来能收获很宝贵的逻辑分析能力,因为很多课程内容,如数学分析,本身就蕴含长逻辑链条的知识脉络,很多推导和证明也都是环环相扣。此外,力学和物理类课程可以帮助大家训练建模能力,也就是“抓主要矛盾”,把一个很复杂的现实问题简化成一个可以解决的数学模型。这些能力会让我们在之后的学习和研究中更有条理、更加深入地考虑问题。

而现行培养方案不可避免的核心问题在于,学习内容有一定难度且抽象程度很高,而课程数量又特别多,只进行课程学习容易停留在抽象层面上沾沾自喜。但学的东西很难并非意味着有多了不起。抽象的理论和具体的问题之间大相径庭、相去甚远,如果只学抽象的东西,回不到具体的例子,其实这些抽象的理论没有什么用;更何况我们学习理论的过程往往只停留在课本和题目上,很少能进行直接应用。这种回到具体例子的训练恐怕需要通过实实在在的大作业和科研训练来完成。

再退一步说,学习很多理论课程并非没有代价——这个代价就是时间成本。现在的大环境下,在本科高年级进行科研训练对于升学和选择研究生导师都是很重要的,如果类似的训练不足,势必对后两者造成影响。而课程的难度和数量,在某种程度上对于学生及早进入科研训练是一种阻碍。

简而言之,工学院的培养方案正在不断迭代中逐渐改进,但问题总归会有。希望大家能主动思考:我希望培养自己的哪些能力?学院的培养方案能在哪些方面帮助我?学院暂时帮不到的方面我该如何自己找办法?


\section{基础课程}
\textbf{\textbf{强调:以下内容为编写团队一己之见,我们相信是具有一定参考价值的,但也鼓励同学们去别处(如认识的老师、学长学姐以及树洞)寻求解答。希望同学们广泛借鉴,敢于探索,闯出一条最适合自己的的学习之路!}}

\subsection{数学课程}
\textbf{1. 数学分析(一)(或:高等数学(I))}

数学分析,也可以称作“微积分”,是一门以微分、积分知识为核心而展开的一系列十分“精确”的分析式课程。课本推荐张筑生老师的三册《数学分析新讲》(黄皮书)。如需要网课,推荐b站上史一蓬老师的数学分析公开课。

笔者对于学好数学分析(一)的建议如下:

(1)注重“极限”“连续”“可导”等基本概念的书写以及“中值定理”“泰勒公式”等定理公式的使用条件,在习题中使用时务必严谨书写,不要写伪证(事实上,如果你能敏锐地发现别人的伪证错在哪里,那么说明你对于概念的掌握已经非常牢靠了)。因此,建议考前做一些系统的整理和记忆;

(2)证明题没有思路时可以试着结合图像或题目条件进行思考,搭建大致思路后再用严谨的语言(如epsilon-delta语言)书写。如果想要一步到位,往往会思路混乱。

(3)书本上一些太过复杂的证明无需完全掌握,只需大概知道思路即可(很多新生容易在这上面花太多时间,导致整体逻辑框架不清晰,重点偏移),且考试一般不会考这类特别难的证明。

(4)掌握几个经典且基本的“反例”,要知道数学分析是很容易因“想当然”而出错的!

(5)多和老师、助教、同学探讨,互相体会对方证明的优点和不足(或者发现对方是伪证doge)。特别强调:利用好习题课!

 \textbf{2. 数学分析(二)(或:高等数学(II))}

来到大一下学期,数学课程难度会陡然上升,你会觉得“上学期学的这么简单但我竟然还喊不会”(这其实说明你水平上涨了很多。当你在学一门课程感觉过程不舒服但并不至于崩溃时,往往是水平上涨最快的时候,不必担心,时间会证明一切。)

数学分析(二)会从可积性开始讲起(上学期最难的部分),然后介绍广义积分、多元微分,多元积分。内容看似简单,实则丰富且变化多端(尤其是进入多元微积分部分),建议课前适当预习(不然容易上课走神就整节课跟不上),并且课后整理,好好做题。同时,需要强调史一蓬老师的教学思想之一:问题不过夜。这是因为数学类课程有很强的前后关联性,未及时解决的问题积累下来会导致后面也听不懂,因此请大家尽量做到“今日事,今日毕”或者“本周事,本周毕”。

学习建议和数学分析(一)大致相同,不过建议投入更多的时间和精力,练习量也要适当增加。

\textbf{3. 线性代数与几何} 

线性代数与几何是一门介绍线性变换、矩阵与行列式运算、向量与几何为核心的课程。不同的老师往往会采用不同的教学顺序,但是整体来说都会大差不差地覆盖以上内容。在多数人眼中要难于数学分析(一)。课本推荐David.C.Lay的《线性代数及其应用》(陈默涵老师所用教材),该书上手容易,框架清晰,习题70\%很简单,很多题目里暗藏玄机(指暗含了很多高等代数的知识,很具有启发性)。如能够读完本书,会得到较大提升。

笔者对于学好线性代数的建议如下:

(1)形成思维导图,知道什么东西对应哪几条性质(很好的例子是“可逆矩阵定理”的24种表述),便于后期融会贯通。

(2)好好完成作业并在作业中总结方法技巧(貌似这是所有课的要求)。很多核心知识(例如如何正交对角化一个矩阵)往往只能在实际操作即做题中掌握,并且算得又快又对。

(3)掌握基本定理的证明,因为证明中相应的思想技巧能帮你更深入了解定理的来龙去脉,而且很容易用在具体的题目当中。

(4)仍然需要多交流探讨。而且线性代数一般会在周末开设“辅导课”,如有兴趣进一步夯实数理基础的可以选择去听。

 \textbf{4. 高等代数}

如果你把线性代数学得特别通透,那么高等代数大概率不在话下。高等代数的核心知识和线性代数基本一样,会额外增添多项式理论、线性空间和空间上的线性变换、若尔当标准型(空间分解定理)等重要内容。课本推荐丘维声老师的《高等代数》(数院用的教材,事实上很多一部分没必要掌握。至于是哪一部分,见仁见智,请听好老师的安排与学长学姐的建议)。

笔者对于学好高等代数的建议如下:

(1)敢于去繁从简。很多证明特别繁杂,技巧性强,往往没精力也没必要掌握,但是同样地,你应该知道它的核心思路。以及,重要的证明还是要会的(例如陈璞老师强调要会简单的归纳法证明)。

(2)从多种角度理解一个定理到底在说什么(简而言之就是把知识串起来),例如:限制线性变换和矩阵对角化的联系、最小多项式的多种求法等。

(3)多做题但不盲目做题。需注意,工院人最核心且基本的技能是计算,因此需要掌握各种跟“算”有关的思想方法和技巧,在此基础上再去了解那些证明(也会相应地变简单一些)。

\textbf{(4)}\textbf{\textbf{(不仅仅属于这一门课程的建议)}}\textbf{\textbf{要}}\textbf{\textbf{学会探索属于自己的学习方法。到大一下学期,你应该尝试着摸索出自己如何从了解到学好一门课程的方法。同时平时切忌偷懒,平时的不了了之会导致后面越来越听不懂,引发“雪崩效应”。}}

\subsection{物理课程}
\textbf{1. 普通物理(I)}

普通物理是工学院学生理论上正式接触的第一门物理类课程,开课时间为大一下学期。普通物理(I)包含力学+电磁学的几乎所有基础内容,且两者往往以期中考试为界限分开。推荐书籍:课本(钟锡华、陈熙谋主编《大学物理通用教程》,共5册书+2册习题解答)就行。

笔者对于学好普通物理(一)的建议如下:

(1)掌握基本概念和书上经典模型(往往课上会强调)。

(2)作业先自己做再和习题解答校对(也可以不校对,等习题课讲评)。如果直接抄,那考试大概率会现出原形,得不偿失。

(3)建议力学部分做适当难度拓展,毕竟力学是整个物理世界的基础,且工学院同学未来学习的内容大多跟力学有关。



\textbf{2. 普通物理(II)}

大二上学期开课,一学期时间内介绍热学、光学和近代物理三部分内容。课本仍为《大学物理通用教程》,但不同老师授课进度和内容选取不尽相同。

笔者认为,相比普物(I),普物(II)内容量更大、更注重理解。因此给出下列学习建议供大家参考:

(1)多阅读课本(有时间可以看看其他教材的讲法),做到理解物理概念和物理过程;

(2)书上的重要公式反复推导直至没有疑点,这能在做习题和考试的过程中帮你节省时间;

(3)抓大放小,关注主干内容,不必事无巨细地希望搞清楚每一句话。课本中很多介绍性内容都在本课的后继物理课程(如热力学与统计物理、电动力学、量子力学、固体物理等)中有严密的理论推导来支撑,但笔者认为学期内较好的状态还是把这门课当普物来学,因为你并非只上这一门课。

(4)适当做题、做完阅读习题解答,这对理解知识和抓住课程重点能起到不错的反向作用。

此外,对普物课来讲,同一课程号的不同班级能互相替代。因此,大家选课前可以广泛参考各渠道的课程评价,对比工院班与数院班、地空班等平行班的课程侧重和讲课方式,选择最适合自己的那个班。当然,想多学一些且有时间、精力的同学也可以选修物理学院开设的“热学”“光学”“近代物理”等课程来替代普物(II)。

\subsection{化学课程}

\subsection{生物课程}



\section{公共课程}

\subsection{计算机基础课程}
        计算机基础课程这里指计算概论和数据结构与算法两门课,分别在大一上和大一下开课。

由于计算机基础课程属于全校必修课,所以由信科老师开课,多个学院同学合上。在选择课程的时候,需要注意不同老师开设课程的语言可能不同,分为C语言和Python。C语言相对更底层更基础,Python相对更简洁开发效率更高,两种语言不分优劣,需要同学们自行选择。此外需要注意,一门计算机课程由课程本体和上机课两部分组成,需要同时选择同一门课的上机课。

在学习计算机课程的过程中,笔者认为需要注意两个方面:基础知识和编码能力。基础知识指理论层面的知识,具体会在考试的“笔试”部分得到考察(即期末考试);编码能力指写代码(coding)能力,需要使用完全正确的逻辑和算法解决问题,具体会在考试的“上机”部分得到考察(学期中某一周考试)。笔试是基础,上机是应用,二者的关系类似于理论与实践,无论哪方面有欠缺都会影响计算机基础课程的学习,在学习过程中不宜偏废。

很多同学没有信息竞赛基础,或者从来没接触过代码,可能会担心不能适应课程,这样的担心是合情合理但不必要的。课程会按照没有基础的同学的学习水平进行教学,无需考虑不能适应的问题。如果实在放心不下可以提前找资料对心仪编程语言的基础语法进行学习。

\subsection{思想政治课的学习}
        思想政治课程如前所述,简称为形策、军理、习概、史纲、思修、马原、毛概、\textbf{\textbf{思政实践*}}。

思政实践比较特殊,我们放在最后介绍。我们首先说明前几门政治课的课程特点。

笼统地讲政治课最终给分一般由以下几部分组成:平时分+论文分+考试分,其中考试分是主体平时分的给法因老师而异,但大多数政治课大多数老师都会设置\textbf{\textbf{小组展示(presentation)}}。小组展示,即同学们在学期初进行分组并领取主题,小组针对课上讲的某一主题进行深入探讨,并在上课时间给出一定时长的课堂展示。

因此一门政治课中同学们的主要任务是:准备小组展示、写论文、准备期末考试。

对于政治课的期末考试,平时功夫固然重要,但是期末复习尤其关键。由于不同老师讲课风格迥异,课程安排也不尽相同,而每门政治课考试又是统考,因此题目不会有很强的个性,这也就意味着题目可被穷举,且有很强的模板性。政治课的最终得分和期末期间背诵的严格程度、完整程度是成强正相关的。

最后,我们介绍思政实践课程。思政实践课程(详情请在practice.pku.edu.cn查看)包括“爱乐传习”“志愿服务”“社会实践”三个模块。“爱乐传习”即一二九合唱比赛;志愿服务如字面所述;社会实践指假期去往全国各地进行实地考察。对于思政实践,同学们唯一需要注意的是时刻关注邮件和学工办通知,不要错过选课时间。

        \subsection{通选课的选择与学习}
        通选课是丰富同学视野,培养通识人才的重要途径。通选课涵盖领域极广,为同学们提供了丰富的课程选择。在选课时,同学们需要注意以下要求:
        
        通选课分为四个系列:I.人类文明及其传统;II.现代社会及其问题;III.艺术与人文;IV.数学、自然与技术。每个系列均包含通识教育核心课、通选课两部分课程,具体课程列表详见《北京大学本科生选课手册》。
       
        通识教育课修读总学分为12学分,具体要求包括:
        \begin{itemize}
            \item 至少修读一门“通识教育核心课”(任意系列)。
            \item 每个课程系列至少修读两学分(通识核心课和通选课均可)。
            \item 原则上不允许以专业课替代通识教育课程学分。
            \item 本院系开设的通识教育课程不计入学生毕业所需的通识教育课程学分。
            \item 建议合理分配,每学期选择一门课程。

        \end{itemize}
        
        通识课程的风格不尽相同。2.3小节所提供渠道中的信息已经非常详细,这里不做赘述。
        
设置通选课的目的就是让同学们接触各种各样不同领域的知识。因此大家根据兴趣(直觉),挑自己喜欢的课程选课、上课就行,没有什么条条框框。
        \subsection{英语分级与选课学习}
所有新生(英语专业学生和留学生除外)入学后须参加英语分级考试,根据考试成绩编入A级、B级、C级和C+级,分别对应8学分、6学分、4学分和2学分的“公共必修课”大学英语课程学分要求,按编入级别修读相应模块的课程。这也是培养方案中总学分是区间而非定值的原因。

课程本身也分为不同级别,即Y,A,B,C,C+。 

各级别应修学分和应修课程模块示意图:

\begin{table}[htbp]
\centering

\begin{tabular}{| l | l | l |}
\hline
入学分级 & 应修学分 & 应修课程 \\
\hline
A & 8 & A级课程4学分+B级课程4学分 \\
\hline
B & 6 & B级课程4学分+C级课程2学分 \\
\hline
C & 4 & C级课程4学分或C级课程2学分加C+级课程2学分 \\
\hline
C+  & 2 & “批判性思维与学术写作”2学分 \\
\hline

\end{tabular}

\end{table}

对于分级考试,同学不必焦虑,发挥正常水平得到符合能力的评级,从而适应对应方案即可。

(详情参考官方文件\href{https://dean.pku.edu.cn/web/rules_info.php?id=66\%EF\%BC\%89}{https://dean.pku.edu.cn/web/rules\_info.php?id=66)}

        \subsection{劳动教育课政策与选课}
    劳动教育课在培养方案中为“32学时”,翻译一下,事实上就是选\textbf{\textbf{一}}门工学院的劳动教育课即可(至于学时到底是多少,教务会协调,总之选一门课就行了)。

工学院目前开设五门劳动教育课,分别是:工学创新实践、金工实习、航空航天金工实习、生物医学工程金工实习、先进制造与机器人实习。详情请根据培养方案要求选课(例如,生物医学工程要求必修生物医学工程金工实习;而理力培养方案并没对实践课作出要求,因此可以任选其中某一门)。当然,选别院系的劳动教育课也未尝不可,不过能否实现认定请咨询教务。

劳动教育课以动手实操为主,能让学生感受到工程技术在实践中的应用,体验感不错。例如:工学创新实践会以某个主题(例如无人机制作)为线索,串联整个学期的学习,让同学们真正实现从零开始,制造作品的体验。

大多数劳动教育课在暑期开课。

        \subsection{公选课以及清华、北外课程介绍}
公选课以各院系面向全校开设的课程为主,可以在预选阶段选择这类课程。但请注意以下两点:第一,根据2024版工学院培养方案,公选课不能计入自主选修课学分,即无法算作毕业学分;第二,公选课中不仅有通识课程,仍有一定比例的专业课,请大家仔细阅读每门课的详细信息。总体来说,公选课给大家提供了一个按自己兴趣选课的广阔平台,大家可以自行探索其中的好课。

下面向大家简要介绍外校(清华大学和北京外国语大学)课程的相关信息:

1.查找开课名单

第一条路径:选课网——培养方案——添加其他课程——课程分类选择“公选课”——开课单位选择“教务部”——查看清华和北外的开放课程

第二条路径:北大教务部官网——直接搜索“清华”或“北外”——查看通知和附件

2.课程特点简介(在此只介绍清华课程,北外开设的小语种课程请自行利用树洞等渠道寻找信息)

清华向我们开放的课程以基础工业训练中心开设的工程实践类课程为主,如“制造工程体验”“现代加工技术与实践”等。这类课程能提供机会供大家学习一些工程上的基本操作、小组合作完成一些作品。课程质量见仁见智;以笔者经历来看,训练中心开设课程质量总体不错,有不少收获。个人认为工学院低年级同学可以选修训练中心的课程来补充实践方面的基本知识。其他院系(如航天航空学院、行健书院等)开设的课程,大家可以根据自己的需求和兴趣进行选择,在学院允许情况下可转为专业课学分。

 3.选课注意事项

清华、北外课程一般比普通课程更晚录入选课系统,选课名额往往很少。选课之后,关注教务的邮件,会详细告诉你怎么操作,包括如何注册清华的课程网、入校方式等等。

清华与北大在课节的时间安排上相去甚远,请大家选课时关注“备注”一栏的实际上课时间,务必提前考虑好通勤时间。举个例子,大家可能会遇到这种情况:北大某课程上课时间为8:00$\sim$9:50,清华某课程上课时间为9:50$\sim$12:15。此时选课网中清华和北大的课程看似没有时间冲突,但实际上没有时间赶到清华的教室(很抱歉,我们暂未找到在清北教学楼之间瞬移的方法)。

 4.学分与成绩互认

结课之后,成绩要到下个学期开学才会由教务部发邮件告知(毕业年级除外),同时告知转学分的方式。学分可以转为通选学分、全校任选学分或专业课学分(详见当年教务部官网通知)。清华课为等级制,不计入绩点,但是等级会出现在成绩单上。此外,转学分需要自己办理,并非默认操作,如果不转学分成绩单上就不会有这一项。

\section{转入/转出工学院}
北京大学向来以“思想自由,兼容并包”为办学理念,因此,在必要的基本规则下,北大为学生提供了十分自由且多元的寻求更好的自我发展的机会,这点很好地体现在了转专业的政策上。当然,转专业必然会有一定的规则,例如需要经历一系列考核选拔、强基计划转专业范围受限(详见1.2.2节)等(必须承认的是,转专业有一定门槛,转入某些院系可能面临很大的竞争,请同学们慎重考量,做好脑力和心理双重准备)。

获取相关信息的主要途径有:
\begin{itemize}
    \item 北京大学教务部官网-“学籍异动”页面

    \item 春季学期开学初全校范围内组织的转系/辅双经验介绍讲座(自行关注相关通知)
    \item 拟转入院系官网、公众号
    \item 有相关经历经验的学长学姐
    \item 北大树洞
\end{itemize}

下图为工学院本科生转出情况及转入情况(数据可能有±1误差):
\begin{figure}[htbp]
    \centering
    \includegraphics[width=0.8\linewidth]{1.4转入转出情况.png}
    \renewcommand{\figurename}{图}
    \caption{工学院本科生转入转出情况}
    \label{fig:enter-label}
\end{figure}

在这里,我们也为大家找到了几位智慧而热心的学长学姐,分享他们有关自己成功转专业的经历和经验。

\subsection{转出工学院经验分享}
\subsubsection{(1)工院$\to$信科$\parallel$作者:一位希望保持匿名的同学,大一平转转入信科信息与计算科学专业(智班)}
\textbf{\textbf{转院考量}}:主要是兴趣以及对于未来的规划。另外建议入学后多了解各个方向,形成更全面的认识后再做决定。本人转信科并非因为觉得工院不好,而是心中向来对计算机抱有更大的热忱,报考北大时就已经坚定地向往去信科学习,只不过因为种种原因被调剂至工院。

\textbf{\textbf{转院流程}}:每年报名及考试时间不尽相同,详情请密切关注院系相关通知(和民间的转信科微信群)。(大概在每年三四月份)信科发布转院系通知后,按照要求填写和提交相关材料。之后教务会通过邮件通知考试时间和安排,按要求参加考试即可。

\textbf{\textbf{课程规划}}:仔细比对信科和工院(自己预期方向的)培养方案,做好两手准备(转信科 or 留工院)。大一上的课大体相似。但是信科学的是计算概论A,工院是计算概论B(根据以往经验,如果计算概论B上80分可以不用重新修读计算概论A);工学院学的是自己开设的数学分析和线性代数/高等代数,信科学的是数院开设的高等数学和线性代数(根据以往经验可以替代)。【以下仅针对部分计算机相关方向的专业,其他例如电子信息等专业会略有不同】大一下的课有部分差别。(根据以往经验)普通物理Ⅰ可以替代专业选修课的物理类课程。信科多出来的两门专业课(程序设计实习和人工智能引论)需要在之后补修,可以选择这其中一门时间不冲突的专业课,减轻之后的补课压力。数据结构与算法B是不能替代数据结构与算法A的,而且数据结构与算法A是在上半学期开课。

\textbf{\textbf{考前准备}}:多刷刷题,打好基础即可。转计算机相关专业的是机考(算法题),转电子信息相关专业的是笔试(物理)。民间转信科微信群里可能会有往年题,可以参考。面试比较中规中矩,不过也可以稍微准备一下。

\textbf{\textbf{住宿调整}}:一般不会换寝室。如果确有需求,据说可以自行找宿管申请。

\textbf{\textbf{心态调整}}:放平心态,即使没考过,大二还有机会(不过到时候课程规划会比较麻烦)。实在没考过也不必太沮丧,工院也是非常值得留下来的。俗话说,三百六十行,行行出状元,真的用心学的话,在哪儿不都一样?

\subsubsection{(2)工院$\to$数院$\parallel$作者:原2022级\ 秦晟,大一降转转入数院数学与应用数学专业}
对于转院,我认为问题最大的可能不是报名流程,而是自己为什么要下这个决定。

1. 首先掂量的是自己想法的“纯度”和“目的性”。一定要避免一时脑子热、跟大流等类似情况。可以旁听几节课,并且注重自己在学这些课程的内心体验,看看是否真如自己所想。

2. 我十分鼓励大家在遇到可能包括转院在内的各种疑惑时,多和老师、家人、同学沟通,之后再看看自己的想法是否有所松动。

3. 关于平转、降转。仁者见仁,智者见智。一年的时间,可以看作节省,也可看作积淀。平转:注意提前选课;降转:较为flexible。

4. 关于流程,教务部每年都于四月中旬发布校本部转专业工作通知与各院系接收人数和考核计划,学生会也会组织转专业、修双、辅修的交流会,有意向同学可以关注北大教务部\&校学生会公众号。

5. 同学们可能会将这类事情看得很重,想着大学专业关系到人生职业、将来发展,就犹豫不知所措。我想说的是,这么想百分之百没问题,但人生纷杂难测,机会很多,遵循当下此时此刻自己的内心想法就好。

祝大家大学本科4(or 5 or 6)年快乐。


\subsection{转入工学院经验分享}
工学院的各位同学可能用不到。但是如果你身边有对转入工学院有兴趣的同学、朋友,可以把这些内容推荐给ta!
\subsubsection{(1)物院$\to$工院$\parallel$作者:2023级\ 李宗远,大一平转转入工学院机器人专业}
北大在大一下和大二下分别有一次转专业机会,我这届转专业在四月初开始报名,报名方式参考北大公示的文件。不过每年时间可能不一样,大家及时关注学院官网上发布的信息以及邮件。至于考核和接收时间就是各个学院自行决定了。我这届是在5.15进行面试,大概一周后工学院公示转专业结果。

每年都北大会发布《XX年校本部各院系转院(系)转专业接收工作具体方案》,里面有各个学院所接收的转专业人数,报名条件,报名方式以及考量方式。在学院官网中可以找到以往的文件。2023和2024年工学院的报名条件都是“在校期间无不及格课程”,听往届学长说以前还对绩点有要求。

23年与24年工学院的转入考量都是只有面试,我这届的面试流程是:学生先进行一分钟的自我介绍(随便说说就行,没多少影响)然后是老师提问。对我而言,老师问的问题是“你这学期选了多少学分的课”“你这学期选的专业课期中考试考了多少分”然后提醒我要提高自己的绩点,我的面试就结束了(狗头)。我的面试时间属于很短的,我个人推测是因为我大一所修的课程和工院的培养方案基本吻合,并且我大一下的几门专业课的期中成绩都不错。老师面试的主要目的是了解学生的情况,未必问具体的专业问题;老师也都很和善,现场的氛围倒是和茶话会挺像。和我一起参加面试另一位同学就被面试了五六分钟。他是一名大二的学生,想要降转工院。但他学习的数学课程是C级,而且在疫情期间选过pf,也退过课。我说的这三条都是面试过程中老师问他的点。

23年工院的招收人数是30人,最后公示接收人数是二十几人。24年工院的招收人数是40人,报名人数21人,接受人数16人,其中选择机器人工程的占了一半。在报名人数仅占接受人数一半的情况下还是刷掉了五个人。个人推测,他们被刷掉的原因可能如下:课程选择与工院培养方案大相径庭或者成绩不好,工院老师担心他转入后跟不上课程进度,还有一种可能是选择转入机器人专业的人太多了,工院老师要考虑机器人系学生的保研竞争问题。

至于平转和降转,大家可以根据自己情况来选择,如果像我一样大一选的课和工院培养方案基本吻合,就可以选择平转,如果专业课有很大不同也可以选择降转,这样可以较好的跟上课程进度。如果课程相差很大还选择了平转,面试老师有可能会问你能不能跟上课程之类的问题。

如果以后工院的转专业考核还和24年一样的话,我的建议是在平时用功夫,多修些相关专业课,把绩点拉高,有什么问题及时询问工院的教务老师和有相关经验的学长学姐,也要记得去官网找最新的我在上文中提到的那几个文件,在大学里信息搜集能力十分重要。至于面试没多少好准备的,当然如果考核方式有所变化就不要按照我的建议来啦。

\textbf{\textbf{课程规划}}\textbf{\textbf{:}}

在前文我也提到,我大一的课程选择与工院的培养方案基本吻合,这在一定程度上帮助我顺利通过工院的面试。同学们可以去北大官网上找一下本科生培养方案(要找最新版的),如果在大一上开始前或者大一下开始前就决定转入工院的话,可以按照工院的培养方案进行选课,这样转过来以后就不用再补课了。或者选一些可以平替工院的课,比如B级及以上数学课。工院23年的培养方案上写着非强基生可以用高数B代替工院的数学分析。其实理学部学生大一的专业课有很多相似之处,比如高数,线代,普物,计概和数算,无非是不同专业难度不同。所以理学部的学生转入工院还是比较轻松的,如果是文科的同学想要转入工院,我建议提前修一些相关的专业课。

非工院学生要在选课的第三阶段才能选工院的专业课,不过工院大一的课程容纳人数很多,名额有很多空余,基本不用抢,在第三阶段选课也没什么影响。

\textbf{\textbf{住宿调整}}\textbf{\textbf{:}}

同学们在转完专业后可自行选择留在原宿舍或者申请换宿舍。留在原宿舍的好处是和舍友已经很熟了,不用再磨合。如果和原宿舍同学相处的不太愉快也可以借此机会换一个宿舍,新宿舍的同学有可能也是工院的,大家平时还可以一起交流讨论专业问题。转宿舍的具体流程要和原学院以及工学院的学工老师沟通。

\textbf{\textbf{心态调整}}\textbf{\textbf{:}}

转入工院的难度要比那些热门的信科,数院小一些,考核也不多,同学们不必焦虑,平时认真学习别挂科,在选课上多花些心思,提前去了解工院的几个专业,确定好自己的目标,转入工院就是水到渠成的事啦。

最后预祝想要转入工院的同学们都成功转进自己心仪的专业,欢迎你们加入工院这个温暖的大家庭!

\subsubsection{(2)生科$\to$工院$\parallel$作者:2022级\ 王\hbox{\scalebox{0.6}[1]{吉}\kern-2pt\scalebox{0.6}[1]{吉}}楷,大一降转转入工学院工程与科学计算方向}
\textbf{\textbf{关于降转和平转}}\textbf{\textbf{:}}

平转即两专业学习无缝连接,降转则需要在新专业多读一年。

如果大家在转专业前后课程重复度较高,或者在转专业前所修的部分专业课,经过目标院系院教务认证可以替代本院培养方案中的课程,个人认为可以直接选择平转。毕竟大一学年以通识教育为主,不同专业的培养方案大相径庭,有心预习目标院系的专业课的话,适应平转的压力并不大。

当然也不必视降转为逃避,如果专业跨度较大,或者关键课程没有修读,完全可以投向降转。多余的一年时间可以更好地适应大学的模式,思考自己真正需要什么,担心荒废时间的话,可以在学有余力且确有热爱的情况下攻读一门双学位,或者提前联系导师进组。(P.S. 亲身体验,唯一的尴尬是说出自己的年级常常伴随着怀疑的目光……

\textbf{\textbf{住宿调整}}\textbf{\textbf{:}}

转专业并不意味着会安排新宿舍。据我了解,除元培PPE和光华未来领导者等少数项目,转入/申请者均搬入了对应的住宿,其他的转专业同学大多仍然维持着原宿舍。搬到同一寝室有助于专业课学习上的互帮互助,也能更快融入新学院。不过就我自己选择了维持原宿舍,因同寝室内共三人转院,算上双学位便在一间寝室内凑出了六门专业,被朋友戏称小元培(bushi),在学术探讨方面为我提供了更多的跨学科体验。
住宿调整方面可以直接前往宿管中心进行询问并申请调整,没有时间限制,随时有空位即可搬。值得一提的是,对于降转的同学,即便当前不换宿舍,在原定的毕业年份时(大三结束后)也需要搬出原宿舍另择宿舍。

\textbf{\textbf{转院流程}}\textbf{\textbf{:}}

0)询问往届学长学姐,或善于搜索,了解往届院系互转的具体政策。我认为应该主要聚焦于一下信息:

\begin{itemize}
    \item 目标院系对转入专业有无限制/本院是否允许转出
    \item 是否对降转/平转有特别要求
    \item 是否需要某门课程必修 \& 该课程成绩要求
    \item 对于绩点要求
    \item 考核方式(面试/笔试)及可能涉及内容
\end{itemize}

在我转专业的那一届,信科仅限平转,设置上机考试且要求需修过计概A;物院转院考试,大一考察力学,大二涉及四大力学;数院则大一大二共用考卷……每年具体政策不同,以上仅为参考,建议尽量掌握当年的相关信息,以便规划自己的选课方案和时间管理。

1)北大教务网搜索转专业专区,下载附件了解转专业申请时间线的大致的时间节点

2)开放申请后,按照要求准备手续材料

3)应对考核 静待结果

……

∞)终章

切莫把转专业当作治愈所有问题的良方,未来规划和学科适应程度确实改变了,可是你所遇到的问题却不会烟消云散,需要你自己去面对和解决。如果因种种原因未能如愿转系,也请放平心态,在北大自由的选课制度和双学位/辅修制度下,找到你的热爱。

\newpage

\section{双学位/辅修建议}
\subsection{双学位/辅修简介}
北京大学非常鼓励学有余力的学生修读双学位、辅修专业,成为复合型人才。

修读辅双的考量\textbf{\textbf{部分}}列举如下:

(1)对某个非主修专业方向很感兴趣,有志于深入钻研的;

(2)认为某个非主修专业方向能够提升自己今后能力、眼界、人脉、竞争力的;

(3)希望以后直接转行至该行业的(注意,强基生跨保其他专业受一定限制,请关注最新通知);

(4)希望所学辅双专业为主修专业赋能的(例如信双进一步提高代码能力,便于工科未来发展等);

(5)暂时没有明确想法,但是学有余力,希望进一步丰富大学学术生活的。

当然,选修辅双的最大挑战就是课业压力会比较大。例如工院强基理力+经济学双学位大约是190学分才能毕业(再次强调每年会变动)。不过学长学姐的经验告诉我们,时间就像海绵里的水,只要愿意挤一挤还是有的(doge)。

关于双学位、辅修的常见核心问题,北大教务部网站有详细解答(路径:教务部官网——学生——双学位/辅修)。

具体的双学位、辅修培养方案也在上述网站中,以文末附件的形式呈现。

同时,每年春季学期开学初,学校会举办转专业/辅双经验介绍讲座,有兴趣的同学请自行关注。

划重点:

(1)双学位要正式报名,辅修不用(因此,辅修无需进行任何手续,只需相应选课、上课,并在毕业的时候跟教务进行相应专业的学位认定即可)。双学位学不下去可以申请退选;如果达到了辅修毕业要求,可在毕业时转为辅修。

(2)双学位课程通过专用通道进行选课,无需投点(直接抽签决定),每学期上限30学分,需要收费(不很贵);辅修课程通过普通选课“跨院系选课阶段”进行正常选课,每学期上限仍为25学分(当然,如果你总GPA>3.7那不管你修没修双还是可以申请超学分选课),不收费。

(3)双学位毕业要修的学分比辅修多(大约45:30),主要体现在专业选修课会多一些。

(4)一门课学分只能认一次。例如:国发院开设的通识课《经济学原理》(3学分)同时也是经济学双学位的必修课。因此可以在毕业时算作双学位的学分,也可以算作主修专业的通识课学分,但只能二选一。

(5)有关毕业证书:例如,工院力学类强基生选择了生物医学工程方向,这时你的专业叫做“理论与应用力学+生物医学工程辅修”,如果再选择双学位(例如国发院经济学双学位),这时你的毕业证书上只能在“生物医学工程辅修”和“经济学双学位”中二选一。

(6)重要事件时间节点:春季学期开学初,会有转系、双学位介绍讲座(全校范围),建议关注。4$\sim$5月转系、双学位报名工作开始进行,请关注相应学院的官网、公众号。

(7)双学位门槛:几乎没有(每个院要求各异,一般性不挂科+GPA正常就行)。并且一般不需要考试选拔。

(8)注意培养方案中的“互斥性”,即特定的A专业和B专业也许不能作为主修和双学位。不过工院在这方面几乎没有受到限制。

(9)注意培养方案是不断更新的。近几年也有新的辅双项目开设(例如元培-人工智能双学位等)。请以最新文件为准。

\subsection{常见问题解答}
\textbf{(1)我到底适不适合学xxx双学位?}

最主要看兴趣是否浓厚长久。一般兴趣能提供动力,会有助于人把课学好。建议在选之前对于培养方案、院系氛围、讲课内容进行细致研究(例如可以旁听一些课程、咨询认识的学长学姐)。实在学不来(理想和现实差距太大)也可以申请退掉双学位,因此不用担心没有退路。

\textbf{(2)学双学位有什么实际作用?}

说实话最大的作用不在于那一张毕业证书。更在于你真正在课堂上学到了什么(说得更玄一点,是学习能力和眼界的提升),是对大学生活的一种充实和拓展。

\textbf{(3)修双学位要提前准备什么吗?}

不必,先把主专业学好了再说。兴趣十分浓厚的学生可以自行找时间阅读相关专业书籍。

 
\subsection{学长学姐经验分享}
\textbf{国家发展研究院-经济学双学位$\parallel$作者:2020级\ 瞿朱毅}

经济学双学位由北京大学国家发展研究院开设,据往年经验,一般要修42学分左右,包含必修课:经济学原理、中级微观经济学、中级宏观经济学、计量经济学和中国经济专题,以及其他选修课。在各个双学位课程中,经济学双学位相对轻松,录取也较为容易。想要未来从事经济学学术科研的同学可以选择相应课程,如做微观理论的可以选学产业组织、博弈论等,如想转商科金融的建议尽早修完财务会计和财务报表分析等。

我当初选学经济学双学位是怀着跑路的心态去的,就是想通过修经济学双学位研究生转往经济学方向,但后来发现经济学学术科研内部内卷严重,工院给分一般导致研究生转到经济学较为困难,故放弃该想法,就把经济学纯当一个兴趣和“白月光“式的存在了。后来强基计划保研政策也公布了禁止转入经管。感觉我最后还是有一些收获,譬如做了助教挣了钱,找到了女朋友(笑),也学会了用经济学的视角分析问题。当然也有遗憾啦,遗憾就是大一没有选上双学位导致有一些必修课放到了紧张的大三去修。优势就是能够多拿一个学位(笑),多认识一些人,多获得一些知识,以及提升绩点,劣势就是在大四还有一堆课,挤占了科研时间。如果说有啥建议的话,那我该说,去努力发掘自己的爱好是一件重要的事情,有些时候经双修的好能够给你从工院低分的挫折里找回信心。

\section{学业导师}

\chapter{学工社团}
    \section{学工简介}
大学生活的一个重要组成部分就是学生工作。

区别于中学,在大学中做学生工作的同学们有更多可以调动的资源,当然这也就意味着更大的责任。如果说“努力学习”是在学术方向上的努力,那么学生工作就是在社会方向上的探索。它教会你怎么服务更广大的群体,怎样和人打交道,怎样做出让大家都满意的工作,所以学生工作是服务他人的方式,更是锻炼自己的机会。

学生工作范畴极大,大体工作领域可以分为:班级,院学生会,院团委,校学生会,校团委,学校各级工作单位。

班级干部工作,主要负责上传下达,跟中学很类似,不做赘述。

接下来简要介绍院级单位和校级单位工作区别。在院级单位,无论是学生会还是团委,我们的工作绝大多数都是围绕学院的师生展开的,优势是我们可以和同院师生更加亲近,能收获更好的关怀,劣势是很少走出学院的“一亩三分地”,活动规模相对较小;在校学生会和校团委的工作则面向全校师生展开,可以结识各个学院的同学老师,参与组织学校的大规模活动,但是可能和学院的同学相对比较生疏。

学生会和团委(院、校同理)的区别,是很多同学都会有的困惑。简要介绍如下:团委举办的活动相对更加正式,强调思政引领;工作模式和风格相对更加规范,组织严明。学生会是党政机关和广大学生联络的纽带,其工作内容和团委有一定程度的交叉,组织架构和团委也较为类似,但是举办的活动相对轻松,工作模式和风格相对自由。

是院团委、院会,还是校团委、校会,如何选择需要同学们自行衡量。

最后我们来说学校各级工作单位的学生助理。很多单位都招收学生助理,例如学生工作办公室助理,招生办公室助理……这些工作的工作时间固定,需要跟老师协调;工作内容是协助老师完成日常工作,对于了解该单位的工作以及相关领域的动向有一定帮助;另外值得一提的是这些工作有固定的薪资酬劳。工作的招募一般在该单位的公众号上发布,有意的同学可以密切关注公众号。

同学们还可以通过辅导员老师,或者通过学院官网coe.pku.edu.cn——教职员工——行政人员,联系各位老师解决学生工作方面的问题。

\newpage

    \section{学生会和团委}
有关校学生会、校团委,上面已经有过一些介绍。详情请关注公众号“北京大学学生会”和“北大团委”,在这里不做赘述。

以下详细介绍工学院学生会和团委的情况。


\subsection{工学院学生会}
学生会是联系学校联系广大同学的桥梁和纽带。工学院学生会的宗旨是\textbf{服务同学、虽小却精、氛围融洽、活动自主}。

在这里你能体验到:
\begin{itemize}
    \item 有趣而温暖的文化。学生会以友爱为先,工作氛围极好,工作压力小,没有恶性竞争,能让人找到很强的归属感。
    \item 丰富而自主的活动。不仅可以体验到各种活动,更可以提供你的点子,为学生会和学院同学创办一些新的活动方案。(相对而言,团委的工作更偏向于正规和纪律性,请大家选择自己合适的工作风格)
    \item 满满当当的成就感。活动能锻炼同学们的能力和眼界,在学生会自主性质的加持下,更能让同学们体会到成功举办活动带来的满满当当的成就感和自我实现感。例如,本手册就是完全由工院学生会学术实践部自主发起和编写的,在撰写的过程中编者也感受到了自我价值的实现!
    \item 优秀而贴心的学长学姐。学生会的学长学姐不仅本身优秀,更是极其愿意帮助低年级的同学们成长向上,在这里你会收获很多生涯发展方面的启发。
    \item 志同道合的同学。如果你急于解决5.3.2小节中提到的第(4)个问题“社恐,想社交但实在不会、不敢,怎么办?”,亦或者你在认为自己满腔的热血和抱负需要寻找伙伴一起实现,那么学生会各部门一定能为你提供这么志同道合的一批人,助你实现自己的理想。
\end{itemize}
学生会现有办公室、外联部、内联部、文体部、宣传部、学术实践部六个职能部门,预计在新生军训、新生训练营期间进行招新,招新对象为工学院全体大一、大二同学。详细介绍请参考公众号“工映青春”上的招新推送(去年的推送为2023.9.1《招新|工学院学生会招新》,敬请期待今年的招新推送)。

随着新工科时代的到来,工学院以及工院学生会的影响力正越来越大,诚挚欢迎大家加入工学院学生会,共创我们的工学新天地!
\subsection{工学院团委}

在工学院团委这个温馨的大家庭里,你可以收获:

\begin{itemize}

    \item 深度参与学院大型活动组织策划的体验
    \item 更早获悉学院一手信息的机会
    \item 较为专业的办公技能锻炼以及组织能力磨炼的机会
    \item 丰富的阅历和视野,同时通过社会实践/志愿服务团体验(例如组织参加支教等志愿服务活动)增长社会技能
    \item 有优秀的学长学姐耐心引路,与志同道合的朋友共同进步的机会
\end{itemize}

学院团委主要包含办公室、组织部、宣传部、文体部、实践部、就业部、青年志愿者协会等部门(可能会稍作调整,请以最新通知为准),招新时间待定,招新对象为工学院全体同学。详细介绍请参考公众号“工映青春”上的招新推送(参考去年的推送为2023.8.31《招新|工学院团委部门招新啦!》,敬请期待今年的招新推送)。

补充:

学生工作不是所有同学一定要参加的(事实上,北大任何活动几乎都没有强制性)。但参加学工,不管是在功利角度还是非功利角度,都会有莫大的收获。请同学们根据自身实际情况,尽力而为,量力而行。

\newpage

    \section{社团简介}
    每学期开学的第二或第三周周末,在百讲与新太阳区域会举行“百团大战”。届时会有上百个社团在场招新,并且在摊位上举办各自的特色小活动。如有兴趣,可以积极参加百团大战并添加相应社团公众号/微信群进行了解。

社团的一些特点是:

1.自治性很强,其活动在团委指导下由学生自行组织;

2.对外开放性不大,导致很多社团甚至不为人所知,看似自娱自乐,实则到底在研发什么“秘密武器”,外人啥也不知道;

3.加入社团没有任何硬性条件考核,完全出于兴趣,很好地体现了北大“思想自由,兼容并包”的精神;

4.在社团里好好干or划水,活动参加or不参加,完全不受强制,完全出于个人意愿,十分自由。因此一般不会大量占用学习时间导致成绩下滑;

5.不同社团大小、氛围差异很明显;

6.社团中也会有负责人等,这也是一种学生工作。 

因此,加不加社团,加几个社团,也是一件“需要你自己衡量”的事。

    \section{入党相关}
加入中国共产党是一件极为严肃的事情。同学们应当在坚定信念后再选择入党,而不是人云亦云,“跟风”入党。

入党有以下几个流程:提交申请书——入党积极分子——发展对象——预备党员——正式党员。以上每次身份的转变都需要经过学习、行动与考察,且周期比较长(总计至少两年多)。

由于流程很长且较为繁琐,这里不赘述。如果读者有志入党,随时可以提交入党申请书给党支部(不必等人统一收取,这是很多人的误区;一般来说给班级辅导员也可以,由其转交),提交申请书后,后续所有流程都会接到通知。

另外提醒:要牢牢记住重要时间点,如何时入团,何日提交申请书,何日被推荐为积极分子等等,这会为以后填写资料节省很多时间。

    \section{其他学生组织与课外活动}
\subsection{志愿服务}
志愿服务是提升人奉献精神和工作能力的良好契机。同时,一定的志愿时长(在“志愿北京”小程序或网站可查看具体时长)还能给综测带来加分,从而给评优评先带来一定优势。因此,花费适当课余时间在志愿服务上,无论从功利还是非功利的角度,都是非常值得的。当然,我们也不鼓励大家去卷过多的志愿时长(注意综测加分48小时封顶),重心仍然应该放在课业学习中。

首先,需要在志愿北京官网https://www.bv2008.cn/上注册账号,并自行记住用户名,密码,志愿者编号等信息。以后在你参加志愿活动后,一般组织方会要求你提供志愿者编号并为你录入相应时长(或者下发纸质证明)。

有的同学苦恼于志愿时长无处可觅,因此在下面介绍几个重要的志愿服务渠道(具体的信息收集仍需要同学们自力更生):
\begin{itemize}
    \item 校级活动:五四青春长跑志愿者(会通知)、北大图书馆志愿者(见图书馆公众号)、返乡宣讲(寒假、暑假都有,会通知)、学生会/团委/青协等学生组织的活动。
    \item 院级活动:比较零散,但是有活动必有志愿服务需求。有意向的同学请咨询学工老师或学长学姐以进入工学院志愿者招募群。
\end{itemize}




\subsection{学校活动}
学校活动数不胜数,种类繁多,只要你想得到,一般就能找得到。

学校层面,除了基本的学工、社团之外,百周年纪念讲堂的表演(详情见百讲公众号)、每逢佳节(中秋、跨年等)的歌会、十佳歌手大赛、各种球类的新生杯/北大杯、餐饮中心的厨艺大赛、图书馆的讲座、地方支教等等各种形式的活动,每个都能带给你别样的体验。活动报名信息获取渠道极为广泛,有意愿的同学请务必时刻关注相关公众号,善用搜索,善于向身边人提问。

学院层面,值得参与的活动也很多。例如由学生会承办的迎新晚会、保研出国经验分享会等;院友之声系列、学生心理类系列等讲座,在此列举不尽。主要关注的信息渠道有:公众号(见手册6.2.1节)、年级群、班级群等。

眼观六路,耳听八方,积极获取信息,相信你一定能找到自己热爱的舞台!

\subsection{一则鼓励}
如果你有好的想法,可以向上建议自行创办有积极意义的学生组织!


\chapter{积极心态}
相信大家既然历经重重考验来到了北大工学院,心理素质都是极强的,各自都有不同的办法对心情、心态进行调节。新生军训及训练营期间,同学们也会接触到一系列的心理讲座,让大家对北大的学习和生活建立良好的心态。在这里,我们针对性地列出一些工学院学生常见的烦恼,并提供一些建议,希望能帮助同学们做好基本心理建设,以积极向上的姿态投入到学习和生活之中。当然,建设强大的内心本就是一个见仁见智的过程,我们希望大家能够吸取其中对自己有用的建议,大胆摒弃行之无效的方法,走自己的道路,过自己的美好人生!

\section{常见问题荟萃}
说明:本部分内容仍待拓展,欢迎大家在我们的问卷中留言或投稿,甚至可以给我们私发消息。我们会选取具有普遍性的问题进行研究和约稿后写入手册(问卷二维码见手册“后记”)。

\subsection{新生面临的核心问题:大学生活与高中生活的不同以及适应方法}


这是一个很宽泛同时又很重要的话题,笔者不敢妄下结论,抛砖引玉,供管中窥豹。

大学生活与高中生活的不同,要从日常生活和工作两方面说起。

在日常生活方面,大学可能是很多同学第一次长时间离开家,随另一个群体生活。与高中生活不同的是,我们得不到高中那样不请自来的关注和指导,因此绝大多数事情需要我们自己去问、自己去摸索,而不能依赖别人主动地悉数灌输给我们。这时候,\textbf{独立性}和\textbf{主动性}就显得格外重要。我们需要学会自己探索,主动求教,为自己的生活负责,这样才可以克服生活中遇到的种种难题。

此外,我们需要面临更真切、更近距离的来自人际关系方面的压力。室友,同学,辅导员老师……处理人际关系的责任不容推脱地落在了我们身上。而由于我们文化环境以及性格、目标的多样性,孤独、矛盾等是不可避免的。笔者想说:人际关系最终的处理水平仍然取决于我们的主观选择。如果他人没有恶意,宽容和体谅总不会出错。

可能有同学已经注意到笔者分类模式是日常生活和“工作”而非“学习”,这是因为人各有志,一些同学不将学术作为未来人生的总目标,这是完全可以理解的。因此我们广义地讲“工作”。

在工作方面,同学们会感觉到前所未有的自由,因为没有人会实时紧盯你的状态,工作和娱乐的权利全在你的手上。如果说高中是轨道,那么大学便是旷野;高中是一幅不容一点瑕疵的工笔画,那么大学便是一张一尘不染的白纸。白纸意味着没有上限,当然也没有下限;这句的前半句是激励,后半句是警告。没有人会阻止我们过放浪形骸的生活,当然也没有人会为我们设定天花板。北大便是这样,资源就摆在那里,不会主动喂到你的嘴边;但是只要你愿意争取,那学校便毫不吝啬地把资源向你倾斜。也许,一些同学已经对“自律”一词有了过犹不及的嫌恶,但是这里笔者仍然将\textbf{“自律”}作为一条极其重要的建议。当然这并不是鼓励打鸡血式地内卷,而是要同学们想清楚自己想要什么,想成为什么样的人,并实时审视自己是离目标越来越近还是渐行渐远。如果适量的惬意休息是海船暂时归港,那么不易察觉的长期放纵便是摧毁海船的暗流与漩涡。

总之,建议有三:一是为自己负责,二是独立与主动,三是宽容与体谅。笔者真诚的期待同学们可以迎来四年无怨无悔的大学生活。


\subsection{学术类}
\textbf{(1)没有竞赛基础怎么适应工学院的学习生活?}

相对其他学院,工学院的竞赛基础相对没有那么夸张,请大家放心“祛魅”。且工学院相对更重视计算而非技巧,因此不必因为无竞赛背景而感到压力。

\textbf{(2)选了数院/物院/信科的课程成绩很差怎么办?}

首先我们要承认其他人在天赋上的优势,并看自己是否能够在意志品质上找补。如果发现成绩不尽如人意,那么建议综合权衡得分和难度再进行选择。诚然,数院物院的课程相对难度更大,抽象程度更高,但不见得就不适合工学院同学进行学习(毕竟力学系学生拿过数学系数分第一名的传说不是盖的),因此具体如何选择需要同学们自己衡量。

\textbf{(3)身边人都开始卷绩点/科研,我怎么办?}

生活不是竞技场。看别人干什么就跟风干什么,最终一定是内卷到疲惫且痛苦的。最好在一开始就做好大概预期,未来是做学生工作,还是搞科研,是留在国内还是去国外,然后再决定自己的生活节奏,不要人云亦云,做无意义的焦虑。当然,也不建议因为没想好上述事情而焦虑,关于前途的考量,很多人是大三大四才定下来的,此所谓“走一步看一步”的智慧。笔者认为大一最主要的任务无非就两件事:适应大学节奏、打好专业基础。真正做好这两件事就很好了。

\subsection{生活类}
\textbf{(1)除了做题,感觉自己其他啥也不会怎么办?}

工学院某位著名的“人生导师”曾经说过:“在北大,千万不要把自己太当回事,也千万不要把自己不当回事;千万不要把别人太当回事,也千万不要把别人不当回事。”无论是感觉自己“啥也不会”,还是感觉别人“啥都会”,事实上都是太把别人当回事,却把自己不当回事了。事实上,北大将教会你的第一件事情就是“祛魅”,让身边许许多多人不再被“神化”,许许多多事情不再“遥不可及”,大家都有自己的优点和缺点,很多也都是经历了高中的“小镇做题”阶段才进来的(必须承认的是,会做题本身也是很大的优点),大家都是一张白纸,等待着你花四年甚至更多时间书写在北大的精彩人生。

北大会让你认识到的第二件事情,就是有梦想就会有实现的机会。第4部分说了,只要想得到的活动一般都能找得到,在北大,资源不会主动找上门,但是如果你积极地去找,一般都是能找到很好的资源进行学习的,同时试错的成本极低(花钱少,不用找关系,地理位置近,教学内容专精……)。因此即便你现在觉得自己“啥也不会”,等过了一个学期,一年,四年,你会发现自己冥冥之中有了巨大的提升。

第三,数院有人说过:“来北大我可能没有成为最好的那个,但至少我知道了最好的是什么样子。”即便发现自己有些事情努力了也做不成,但是你已经去尝试过,也见过最好的事物是怎么样的,这本身就是一个眼界和能力的巨大提升,对未来帮助很大,同时,在未来你仍然有机会再次挑战它。记住,你才18岁。


\textbf{(2)学工、活动占据过多时间,导致成绩下降怎么办?}

工学院的一位著名学长曾经说过:“大学就是在不同赛道上成就自我。”大学赛道不再单一化,成绩不再是成功的唯一衡量标准。在成绩基本盘hold住的情况下(指至少不能挂科,绩点还说得过去),有的同学选择深入钻研课程,力争更高的成绩,为将来保研出国和深入研究打下基础;有的同学选择将大部分时间投身科研,追求知识的深度,早早开始挑战真正有希望为人类世界带来GDP的业界技术难题;有的同学选择学习辅双或选择别院系的课程,拓展自己的学术兴趣、志趣,在知识的广度上实现自我价值;有的同学选择投入学工,发挥自己的社交属性,和老师、同学们深入交流,并在工作中锤炼实用技能,为将来的工作能力奠基;有的同学不断寻找实习,为将来进入业界打下坚实的经验基础,同时凭着实习的工资实现小小的“财务自由”;有的同学选择多多参与文体活动,在活动中绽放自我,结交朋友,成为舞台上最闪亮的星星;有的人拓展自己的爱好,坚持养成一个个好习惯,不在外表上宣扬自己的强大,内核却永远在悄悄变强,期待着惊艳大家的那一天......以上种种,都是“赛道思维”很好的体现,只要是实现了自己的价值就有意义,不能说哪条赛道就一定比另一条好,正所谓“不管黑猫白猫,能抓到老鼠的就是好猫。”

因此,如果出现了上述类似的问题,笔者的建议是,你需要衡量自己心目中各条赛道的优先级,坚守课业成绩基本盘不动摇的同时,摒弃“成绩为王”的高中生思维定势,选择最能实现自我的道路坚定地发展。

\textbf{(3)和家长、同学、恋人关系出现危机,影响正常生活,怎么调整心情?}

\begin{itemize}
    \item 直面并解决问题。建议直接当面或电话沟通,微信聊天往往可能引起歧义。
    \item 吐槽。跟信得过的老师、好友吐槽,大学新增的方法是树洞吐槽。
    \item 事情扔一边先冷静。如果不是很急于解决,先不管这件事情,正常学习生活一段时间,冷静之后再回想。
    \item 咨询。当影响很大时,千万不要藏着掖着,要敢于寻求专业的心理援助(事实上,心理咨询中心不能看作是“治病”的地方,更多时候它是一个可以放心跟人吐槽,同时有人帮你理性分析、解决问题的地方)。
\end{itemize}

\textbf{(4)社恐,想社交但实在不会、不敢,怎么办?}

社交是大学的必修课。首先,不要用标签框定自己(如MBTI),要知道社交属性到大学是会变化的(例如有的同学由i转e)。其次,不同性格的人各有舒适区,具体不再赘述,请大家自行感受,并学会悦纳自我。最后还要意识到一件事情是,社恐和社牛并没有明显的界限划分,大多数人其实是处于中间的灰色地带的。

认识到以上几点之后,相信“社恐”的同学就不会再过多产生社交焦虑了。如果实在觉得自己需要社交的机会,不妨先从大胆参加活动、多多认识志同道合的朋友开始吧。当然,最终能不能成为朋友很多时候是看缘分,强求不来。

最后提醒大家社交把握尺度,拥有自己的判断力,不要让盲目社交影响了你健康的生活和正常的自我发展。

\section{寻求心理帮助}
学校为我们开设了完备的心理咨询服务,心理咨询中心会开设各类主题活动,可以关注公众号“北大学生心理健康教育与咨询中心”以参加相应活动。如果遇到不能解决的困难,可以拨打心理咨询中心电话(前台62760852 ;24小时心理援助热线62760521)。

请读者牢记,心理障碍或创伤和一般的创伤是一样的,是不可抗因素,不显示主观意志的强弱。当我们觉得难以支撑、心力交瘁时,我们要知道这不是因为我们脆弱或敏感。因此,请积极地向外界寻求帮助,这并不羞耻。心理障碍可以被克服,我们终究会迎来拨云见日之时。


\chapter{日常生活}

\section{院系大楼}
自去年新奥工学大楼建成以来,工学院各办公室陆续向燕东新园\footnote{即工学院、物理学院、城市与环境学院等院系所在的区域}迁移,未来工学院在校内的分布将集中在工学大院的三栋楼内,即力学楼、工学院1号楼和2号楼(新奥工学大楼)。

\begin{figure}[htbp]
    \centering
    \includegraphics[width=0.6\linewidth]{6.1工学大院地形.jpeg}
    \renewcommand{\figurename}{图}
    \caption{工学院主要建筑分布及通勤路径}
    \label{fig:enter-label}
\end{figure}

教务办公室和学生工作办公室均已搬入新奥工学大楼,分别位于2005室和2012室,负责其他事务的办公室位置也可进入工学院官网——办公服务——工学院办公室一栏查看。

在手册写作的2024年7月,老师们的办公室仍在搬入新大楼的过程中,具体完成时间暂不知晓;且并非所有老师的办公室都会搬入新大楼,有些老师仍会留在燕园主校区内。同学们如想与老师面谈最好提前确认办公室地点。

工学大院的三栋主要建筑内都有会议室。大多数情况下,学院内的大规模会议在这些地方举行:力学楼为434室,1号楼为210室,2号楼(新奥大楼)为3004室。

此外,在燕园主校区,燕南园60号也是工学院部分部门的办公地点,院子东北角立有写着“北京大学工学院”字样的石碑,是工学院新生打卡圣地。同学们通勤穿过燕南园时经常能路过60号院。

\section{信息渠道}
在熟悉北京大学生活的过程中,同学们会逐渐意识到大学阶段与高中有一个差别很大的地方——大学期间诸多信息都需要学生自己主动去了解,而不会打包好送到学生面前。因此,高效地获取和筛选信息的能力可以尽早培养,一定程度上能预防“信息差焦虑”的出现。

在这里我们列出最常用的一些,便于新入学的同学们熟悉工具的使用,提高主动搜集、整理信息的能力。

\subsection{自助}
\textbf{(1)网站类}

1. 北大信息门户portal.pku.edu.cn(集成了北大日常生活需要的几乎80\%的工具和信息。如数字校园卡/校历/课程相关/成绩相关/体育场馆预约/访客预约/空闲教室/畅行清华等等。有网页版,也有手机APP,但手机版适配做得不太好。)

2. 北大教务部官网dean.pku.edu.cn(可查询关于学业、课程的绝大多数相关信息)

3. 北大树洞treehole.pku.edu.cn(匿名聊天平台,可以查询非常多民间信息,同时也有很多没有多少价值的内容。务必注意树洞言论也需要遵守底线思维。)

4. 北大教学网course.pku.edu.cn查看课程内容和获取课程资料的平台(选课后使用)

5. 北大邮箱mail.stu.pku.edu.cn 有时会有重要邮件,需要定期查收.

6. 北大工学院官网coe.pku.edu.cn(与工学院相关的信息/文件/新闻大多可以在官网找到。可以查询工学院办学情况、师资队伍等信息,网站上也有老师的联系方式等)

7. 北大空间(微信小程序,预约研讨教室。可预约地点为二教五楼小教室、四教教室、地学楼玻璃研讨间等)

8. 博雅学堂预约系统byxt.pku.edu.cn(预约老化学楼一楼的讨论教室。博雅学堂理论上仅供强基同学使用。)

\textbf{(2)公众号与小程序}

1. 北京大学官方公众号

2. 工映青春(北大工学院学生活动信息平台,会发布一些重要通知)

3. P大CoE教务(北大工院教务公众号,会发布学业、课程相关重要通知)

4. 北大体育、PKU体委(注意关注85km跑的相关通知)

5. 赛艇先生(考试复习资料、往年题集成平台,\textbf{强烈推荐})

6. 青春北京(与团组织相关事宜有关)

7. 北京大学百周年纪念讲堂(电影票及演出等信息)

8. 北大团委(学校团委活动通知,如中秋晚会)

9. 北大就业(就业信息)

10. 北大工学(工学官号,偏学术)

11. 北大与世界(对外交流等信息)

12. 北京大学学生发展支持(辅导课等学业支持活动)

13. 北京大学学生会(校学生会活动通知,如十佳歌手比赛等)

14. 北京大学图书馆(小程序,用于图书馆馆藏内容的检索、图书馆云打印服务等)

15. 北大餐饮中心官方资讯(用于了解学校里哪里有好吃的doge)

16. 北大空间(小程序,用于预约北大教室用来自习、活动、准备pre等)

17. 志愿北京(小程序,用于查看自己的志愿服务学时)

18. 北大文创(小程序,用于文创产品的线上购买、邮寄)

\textbf{(3)软件类}

1. 北京大学APP(同北大门户)

2. bilibili(无数宝藏课程的荟萃地,同时也是娱乐的天堂)

3. MATLAB(工学院必备,但大一用处不很广)

4. 各类IDE(Integrated Development Environment,集成开发环境;用于编写代码,如PyCharm,Spider,Dev C++,Visual Studio Code等,具体听从计算机课程老师和助教的安排,再自行比较择优)

5. Notability、GoodNotes等(pad端记课堂笔记软件,使用时\textbf{注意及时备份})

\textbf{(4)AI工具}

ChatGPT、Kimi智能助手、文心一言、讯飞星火等(功能强大,可以进行文案和代码撰写、修改等实用功能,但切忌盲目搬运)

\subsection{他助}
\begin{itemize}
    \item 基本生活求助:见6.3节

    \item 高级生活求助:学工老师、辅导员

    \item 课程内容求助:授课老师、助教、同学

    \item 学业大局求助:导师、班主任(事实上绝大多数老师都很乐意解决学生学业大局上的疑难)、学长学姐

\end{itemize}


\subsection{基本生活需求}
下面列举一些机构的联系电话,帮助同学们解决困难和需求。此部分在新生讲座也会有介绍,请大家记录好相关信息。
\begin{itemize}
    \item 保卫部:62755110 (生病受伤等不能行动情况下可以打保卫部电话被接到校医院)

    \item 心理咨询中心:前台62760852 ,24小时心理援助热线62760521

    \item 校医院:62759011(接受校内外人员的咨询预约),62751919(24小时急诊)

    \item 北大周边部分酒店:勺园62752218,62757361,62752200;中关新园62755236;资源宾馆62757199(如有亲戚朋友来访,可以安排住在北大附近,学生卡可以打折)

\end{itemize}

\backmatter

\chapter*{后记}
这篇后记写得相对自由一些,内容也并不紧凑,是我想到哪写到哪的产物;不想看的读者也大可直接合上本手册。当然,我们非常感谢各位耐心的读者愿意读到这一页。

最初产生编写手册的想法,是在2023年9月22日的学术部例会上,那时项目的名称还拟定为“Engineering DIY”。当时由于学术部三位成员都刚加入部门,没有太多工作经验,对这项可能带来巨大精力消耗的提议一致采取了推迟的决定,先着手进行课程专访等工作。

想法仍然存放在我的心中。只是对于那时的我,还没有十足的把握承担起这份工作,所以产生过“等我本科快结束时再写”的想法。

这之后的故事,前言里已经讲了一些。机缘巧合之下,编写手册的任务被许多人合力推上了发射的轨道。

5月中旬,学术部三人组根据自己的经历和观察初步确立了手册的框架,梳理了各部分讨论的核心问题。繁忙的期末考试结束之时,手册的撰写工作逐步展开;这中间存在细水长流的阶段,也有过集中攻关的时刻。进入8月,在接受了审核与细节核实之后,经过多天的封面制作和排版调整,这份5万余字的《工学院本科生生存手册(新生版)》终于基本完成。在此感谢所有为本手册做出贡献的老师们和同学们。

%人名
\textbf{以下是参与《工学院本科生生存手册(新生版)》制作的全体人员名单:}

\textbf{主编:赵丹枫\hspace{6pt}倪\hspace{11pt}昊\hspace{6pt}武昱达}

\textbf{封面美工:周文硕\hspace{6pt}田浩远}

\textbf{文字作者:}

\textbf{2020级本科\hspace{20pt}王一诺\hspace{6pt}瞿朱毅}

\textbf{2021级本科\hspace{20pt}刘家豪\hspace{6pt}李一川\hspace{6pt}吴秉宪\hspace{6pt}杨铮昊\hspace{6pt}钱骏飞\hspace{6pt}谢谊锋}

\textbf{2022级本科\hspace{20pt}王\hbox{\scalebox{0.6}[1]{吉}\kern-2pt\scalebox{0.6}[1]{吉}}楷\hspace{6pt}付\hspace{11pt}杨\hspace{6pt}李昆泰\hspace{6pt}范文琳\hspace{6pt}赵丹枫\hspace{6pt}郭\hspace{11pt}祺}

\textbf{\hspace{82pt}秦\hspace{11pt}晟\hspace{6pt}曹林博\hspace{6pt}曾帅鹏程}

\textbf{2023级本科\hspace{20pt}李宗远\hspace{6pt}武昱达\hspace{6pt}倪\hspace{11pt}昊}

\textbf{2024级本科\hspace{20pt}古冠群\hspace{6pt}郑雅文\hspace{6pt}郭濠源}

\textbf{特别鸣谢:刘牧时\hspace{6pt}刘\hspace{11pt}威\hspace{6pt}王显宁\hspace{6pt}谢谊锋\hspace{6pt}吴秉宪\hspace{6pt}陈欣玮\hspace{6pt}吕浩鑫}

\textbf{监制:李张鑫}

\textbf{以及几位希望保持匿名的同学,再次感谢所有老师们和同学们的付出!}
 

由于编者水平所限,想必会有很多内容上的疏漏和排版失误,在此敬请读者指正,任何关于《工学院本科生生存手册》的纠错、建议、以及投稿都欢迎通过下面的二维码与我们联系。提前感谢您为本手册作出的贡献。
\begin{figure}[htbp]
    \centering
    \includegraphics[width=0.2\linewidth]{后记二维码.jpeg}
\end{figure}

为了能让2024级同学们尽早看到这份手册,本着精简人员、缩短战线的原则,新生版的内容重点主要来源于对三位主编个人经验和并不充分的身边采样的拟合,很难全部覆盖新生的需求和疑问。因此,我们将在接下来的学期通过发布调查问卷等方式进一步了解大家的需求,并不断对本手册进行修订,目前计划如下:

2024$\sim$2025学年第一学期,通过调查、约稿等方式补齐面向较高年级的“专业发展”“学术科研”“升学规划”三部分内容,形成完整版手册;勘误、内容细节的修订与精简。

更新工作相应信息会及时在树洞\#6537956发布。请有兴趣的同学们关注本洞,并以洞主发布的相关信息为准。

\textbf{版本更新:}

V1.0\ 新生版 定稿日期:2024年8月16日

\begin{flushright}
    赵丹枫

    2024年8月
\end{flushright}
\chapter*{结语}
雅思贝尔斯讲:“教育是一棵树摇动另一棵树,一朵云推动另一朵云,一个灵魂唤醒另一个灵魂。”笔者自知身为“先行者”,决不能耳提面命,不能高高在上,不能不容置疑。诚望能将自己的经验、自己走过的道路汇编成册,抛砖引玉式地为读者提供参考,让读者更快更好地“上手”大学生活。

洋洋洒洒写到结尾,笔者深知前文中仍有片面的、主观的认识,若对读者造成困扰,还望见谅。

人生是旷野,不是轨道。我们真诚地祝愿读者不囿于一隅,能够在这片旷野里恣意奔跑,在燕园中纵情挥洒青春年华。

\begin{flushright}
    武昱达

    2024年8月
\end{flushright}
\end{document}